\chapter{Netgenerering} \label{netinddeling}

Målet med dette kapitel er at præsentere en netinddelings 
metode, der giver asymptotiske optimale netinddelinger 
(indenfor klassen af netinddelinger med rektangulære 
elementer). Metoden bygger på begrebet net tætheds funktion 
og blev indført af Babu\v{s}ka og Gui i~\cite{ba-gui}. Metoden er 
senere videreudviklet af Hugger i artiklerne~\cite{hugger-1}, 
\cite{hugger-3}, \cite{hugger-2}, \cite{hugger-4}, \cite{hugger-pol}
og senest af Babu\v{s}ka, Hugger et al i \cite{hugger-net}. Med net 
tætheds funktioner kan man beskrive følger af netinddelinger
ved brug af simple midler, og man undgår dermed de 
komplicerede datastrukturer, der ofte karakteriserer 
netinddelings algoritmer. Ønsket om simple metoder til at 
beskrive netinddelinger på udspringer fra parametriserede
problemer, hvor man med fordel ofte kan genanvende tidligere
netinddelinger. Det vil fremgå af teorien nedenfor, at 
metoden med fordel også kan anvendes, selvom man kun skal 
løse et enkelt problem. Netinddelings algoritmen er ment som 
en ``one shot'' metode, der på baggrund af informationer fra 
net tætheds funktionen og den bruger bestemte fejltolerance 
forsøger at konstruere en netinddeling, hvor fejlen i den
tilhørende endelige element løsning efter første gennemløb af 
algoritmen er mindre end den givne fejltolerance. Mislykkes 
dette, vil metoden iterere.

\section{Sammenhæng mellem net tætheds funk\-ti\-on\-er og følger af net} 
Før vi definerer net tætheds funktioner, har vi brug for følgende
\begin{definition}
Lad $\T=\{\Box_j\}_{j=1}^m$ være et endeligt element net. Vi
siger da, at $\T$ er en optimal endelig element netinddeling
indenfor en given klasse af netinddelinger, såfremt fejlen
i den tilhørende endelige element løsning er mindre end en 
given fejl\-to\-le\-ran\-ce, og såfremt de beregningsmæssige 
omkostninger ved at finde denne løsning er minimale. Minimum er 
her taget over alle netinddelinger indenfor den givne klasse.
\end{definition}

\begin{definition} \label{defcfin}
Lad $\dom$ være en åben sammenhængende mængde i $\R^n$. Ved mængden af
endeligt ikke-glatte funktioner defineret på $\dom$, vil vi forstå
mængden af reelle funktioner defineret på $\dom$, som er uendeligt ofte
differentiable (${\mathcal C}^\infty$) i $\dom$ undtaget på et antal
delmængder af $\dom$. Om disse mængder (hvor funktionen ikke tilhører
${\mathcal C}^\infty$) skal der gælde, at antallet af disse i enhver
begrænset delmængde af $\dom$ er endeligt. Desuden antages enhver af
disse mængder at være glat og endelig i den forstand, at en sådan
mængde enten er et punkt eller en $k$-dimensional ${\mathcal
C}^\infty$ undermangfoldighed af $\R^n$ med endeligt $k$-volumen, hvor
$k\in\{ 1,\ldots ,n-1\}$. Vi vil betegne mængden af endeligt
ikke-glatte funktioner defineret på $\dom$ med $\Cfin$.
\end{definition}
\begin{remark}
Er $\dom$ fra definition~\ref{defcfin} begrænset, er der kun endeligt
mange ``singularitets'' mængder.
\end{remark}
\begin{remark}
Funktionerne i $\Cfin$ er tilstrækkelige til at beskrive stort set
alle praktiske fænomener indenfor anvendt matematik.
\end{remark}
Ved brug af funktionerne i $\Cfin$ kan vi nu definere net tætheds
funktioner som
\begin{definition}  \label{density}
En net tætheds funktion (eng. mesh density function) defineret på
en åben sammenhængende mængde $\dom\subset\R^n$, er en næsten overalt
positiv (eventuelt ubegrænset) funktion $\D$ tilhørende $\Cfin$, som opfylder
\begin{equation}
  \int_\dom \D(x)\, dx = 1.
\end{equation}
Mængden af net tætheds funktioner betegnes med $D_\dom$.
\end{definition}
\begin{definition}
Lad $\re$ være et element i en netinddeling, og lad $\D$ være en net
tætheds funktion. Ved tætheden af $\re$ forstås værdien af 
$\int_\re \D (x)\, dx$.
\end{definition}
Net tætheds funktioner blev indført af Babu\v{s}ka og Gui i
\cite{ba-gui}. Deres ide var at konstruere netinddelinger på en sådan måde,
at tætheden i ethvert element i netinddeling\-en er den samme.
Principielt kunne man arbejde med netinddelinger, hvor tætheden varierer
fra element til element, men kravet er praktisk set fra en
implementations vinkel. Ved hjælp af ovenstående definition kan dette
krav udtrykkes som
\begin{equation} \label{ideal-con}
  \int_{\re_j} \D (x)dx = 1/N_a \quad \text{for $j=1,\ldots,N_a$,}
\end{equation} 
for en netinddeling $\T=\{\re_j\}_{j=1}^{N_a}$. I praksis vil man kun 
kræve, at \eqref{ideal-con} gælder approksimativt.

Bemærk at \eqref{ideal-con} beskriver en hel følge af netinddelinger,
en netinddeling for hvert positivt tal $N_a$. Net tætheds funktioner
kan altså bruges til at beskrive følger af net og ikke kun et enkelt
net. Artiklen \cite{hugger-3} indeholder resultater, der i det
asymptotiske tilfælde og under passende forudsætninger viser, at
vælges net tætheds funktionen på
fornuftig vis, da vil normen af fejlen i den endelige element løsning være
minimal. De følger af netinddelinger vi fremover skal arbejde med
antages at have den egenskab, at antallet af elementer i netinddelingen
er ikke-aftagende og divergerer mod uendelig. En mere formel
definition, der også giver os den nødvendige notation, lyder
\begin{definition}  \label{nondec} 
Lad $\dom$ være en åben sammenhængende mængde i $\R^n$, og lad
$\{\T_m\}_{m=1}^{\infty}$ være en vilkårlig følge af endelige element
net på $\dom$. For ethvert positivt tal $m$ betegner $N_{a,m}$ antallet af
elementer i $\T_m$. Mængden af ikke-aftagende følger af endelige
element net, betegnet med  $P_\dom$, defineres da som de følger af
endelige element net $\{\T_m\}_{m=1}^{\infty}$, hvor den tilhørende følge
$\{N_{a,m}\}_{m=1}^{\infty}$ er ikke-aftagende og divergerer mod uendelig.   
\end{definition}

Følgende definition giver en sammenhæng mellem net tætheds
funktioner og ikke-aftagende følger af endelige element net.

\begin{definition}
Lad $D_\dom$ og $P_\dom$ være som i definition~\ref{density} og
\ref{nondec}, og lad $\dom$ være en åben sammenhængende mængde i
$\R^n$. En net tætheds operator $A$ (eng. mesh density operator) 
defineret på $\dom$ er en invertibel afbildning
\begin{equation}
  A:\D\in D_\dom \rightarrow \{\T_m\}_{m=1}^{\infty} \in R_A \subset
  P_\dom \ ,
\end{equation}
som afbilder mængden $D_\dom$ af alle net tætheds funktioner på en
delmængde $R_A$ (billedet af $A$) af mængden $P_\dom$ af alle
ikke-aftagende følger af endelige element net.
\end{definition}

At der rent faktisk eksisterer net tætheds operatorer er ikke
oplagt. I det endi\-men\-sio\-na\-le tilfælde er det let at bevise, se fx
\cite{hugger-2}. Et bevis i det $n$-dimensioneale tilfælde er mere
kompliceret, idet \eqref{ideal-con} ikke indeholder informationer nok
til at bestemme en netinddeling (eller en følge af netinddelinger)
entydigt. Udfra \eqref{ideal-con} kan vi kun få oplysninger om
elementernes tæthed og ikke om deres geometriske form. Dette
problem kan løses på to forskellige måder. En fremgangsmåde er at tilføje
betingelser til \eqref{ideal-con}, der specificerer elementernes geometriske
form. Alternativt kan man vælge kun at benytte en bestemt type af
netinddelinger med foruddefinerede geometriske former for
elementerne. Vi skal her forfølge den sidstnævnte ide.

Før vi definerer tilladte netinddelinger og elementer, er det
nødvendigt at præcisere, hvilken form vi vil tillade for domænet for
det betragtede problem. Tilladte domæner vil blive benævnt
$\beta$-domæner. Ideen er at tillade domæner i $\R^n$, der kan
inddeles på en sådan måde, at hver delmængde i inddelingen kan
afbildes bijektivt på en enheds $n$-kube $\re_n=[0,1]^n$. Denne afbildning
skal desuden bevare glatheden i den endelige element løsning i de
enkelte elementer samt over randene mellem elementerne. Figur~\ref{exbetadomains} 
indeholder et par eksempler på tilladte domæner. Ideerne i de følgende
definitioner er udviklet i \cite{ba-mil} hhv. \cite{hugger-2}. 
\setlength{\unitlength}{1mm}
\begin{figure}[htb]
\begin{center}
\begin{picture}(100,40)(0,0)
\put(0,2.5){\begin{picture}(40,35)(0,0)
\put(0,0){\line(1,0){40}}
\put(0,0){\line(3,5){20}}
\put(40,0){\line(-3,5){20}}
\dashline{2}(20,12.5)(10,16.67)
\dashline{2}(20,12.5)(30,16.67)
\dashline{2}(20,12.5)(20,0)
\end{picture}
}
\put(60,0){\begin{picture}(40,40)(0,0)
\put(0,0){\line(1,0){40}}
\put(0,0){\line(0,1){40}}
\put(0,40){\line(1,0){40}}
\put(40,0){\line(0,1){15}}
\put(40,40){\line(0,-1){15}}
\put(20,20){\line(4,1){20}}
\put(20,20){\line(4,-1){20}}
\dashline{2}(20,20)(0,20)
\dashline{2}(20,20)(20,0)
\dashline{2}(20,20)(20,40)
\end{picture}
}
\end{picture}
\end{center}
\caption{Eksempler på $\beta$-domæner\label{exbetadomains}}
\end{figure}
En definition af $\beta$-domæner kræver lidt notation vedrørende enheds $n$-kuber.
\begin{definition} \label{rside}
Ved en afsluttet $r$-side i enheds $n$-kuben $\re_n$ for
$r=0,\ldots,n$ forstås en delmængde af $\re_n$ med $r$ frie
koordinater, og hvor de $n-r$ resterende koordinater enten er $0$
eller $1$.
\end{definition} 
Med denne definition er $0$-siderne i $\re_n$ hjørnepunkterne, og $1$-siderne
er kanterne. Vi kan nu give en definition af $\beta$-domæner.
\begin{definition} \label{betadomain}
En åben sammenhængende mængde $\dom\subset\R^n$ kaldes et
$\beta$-do\-mæ\-ne, såfremt følgende betingelser er opfyldt
\begin{itemize}
  \item $\dom$ kan inddeles i et endeligt antal $l\geq 1$ af parvis
        disjunkte, åbne, sammenhængende delmængder $\dom_d\subset\dom , \
        d\in\{1,\ldots,l\}$ på en sådan måde, at $\overline{\dom} =
        \cup_{d=1}^{l}\overline{\dom}_d$. 
  \item Der findes $l$ invertible ${\mathcal C}^\infty$ afbildninger
        $\beta_d, \ d\in\{1,\ldots,l\}$, hvor afslutningen af
        $\beta_d(\dom_d), d\in\{1,\ldots,l\}$ er enheds $n$-kuben $\re_n$.
  \item For to vilkårlige $r$-sider $(r\in\{ 1,\ldots ,n\})$ i to 
        forskellige delmængder $\dom_e$ og $\dom_f$ i inddelingen af $\dom$,
        skal der gælde, at fællesmængden for $r$-siderne enten er tom eller en
        hel $s$-side for et passende $s\leq r$.
  \item Afbildningerne $\beta_d, \ d\in\{1,\ldots,l\}$ skal bevare en
        eventuel glathed over randene mellem delmængder i inddelingen af $\dom$.
  \item Lad $u_d, \ d\in\{1,\ldots,l\}$ være funktioner tilhørende
        ${\mathcal C}^{k_{\max}}(\re_n)$ defineret på enheds $n$-kuben. Her er 
        $k_{\max}$ det størst mulige tal som opfylder, at ${\mathcal C}^{k_{\max}}$ er
        et underrum af testrummet for den betragtede endelige element metode.
        Antag er der for to punkter $y_1$, $y_2\in\re_n$, findes $e$,
        $f\in\{ 1, \ldots ,l\}$ med $e\not = f$, så $\beta_e (y_1)=\beta_f (y_2)$. 
        Da skal $Du_e(y_1)=Du_f(y_2)$ for alle afledede $D$ af orden mindre end eller
        lig med $k_{\max}$. Desuden skal funktionen $U$ givet ved
        $U|_{\dom_d}=u_d \circ \beta_d$, $d=1,\ldots,l$ være entydig
        defineret i hele $\dom$ og tilhøre ${\mathcal C}^{k_{\max}}(\dom)$.  
\end{itemize}
\end{definition}
For $\beta$-domæner skal vi definere en bestemt type af netinddelinger
også kaldet $\beta$-netinddelinger.
\begin{definition} \label{betamesh}
Lad $\dom\subset\R^n$ være et $\beta$-domæne, og lad notationen være
som i definition \ref{betadomain}. Lad $\T_d
=\{\re_{j,d}\}_{j=1}^{N_{a,d}}$ være et endelig element net over
$\dom_d$ for $d\in\{1,\ldots,l\}$, og lad $\T=\cup_{d=1}^l \T_d$ være
netinddelingen af $\dom$ bestående af alle ele\-menterne i delnettene.
Da defineres $\beta$-netinddelingen $\T_\beta$ hørende til $\T$ som
$l$-tuplen $(\T_{1,\beta},\ldots ,\T_{l,\beta})$ af netinddelinger
over enheds $n$-kuben, hvor
$\T_{d,\beta}=\{\beta_d(\re_{j,d})\}_{j=1}^{N_{a,d}}$ for $d=1.\ldots
,l$. Pr. definition er elementerne i $\T_\beta$ mængden bestående af
$\beta_d(\re_{j,d})$ for $j=1,\ldots ,N_{a,d}$ og $d=1,\ldots ,l$.
$\beta$-netinddelingen $\T^{\ast}=(\beta_1(\overline{\dom}_1),\ldots
,\beta_l(\overline{\dom}_l))$ kaldes for reference
$\beta$-netinddelingen af $\dom$ (hørende til inddelingen
$\{\dom_d\}_{d=1}^l$ af $\dom$ og afbildningerne $\{\beta_d\}_{d=1}^l$).  
\end{definition}
\begin{definition}
En delmængde $\re$ af $\R^n$ siges at være $n$-firkantet (eng.
$n$-quad\-ri\-la\-te\-ral), såfremt der findes en lineær afbildning $L:\R^n
\rightarrow \R^n$ så $L(\re_n)=\re$, hvor $\re_n$ er $n$-enheds kuben.
Hvis kanterne (se definition~\ref{rside}) i $\re$ har samme længde, og
alle vinkler er rette, kaldes $\re$ for $n$-kubisk.
\end{definition}
\begin{definition}
Lad $\dom\subset\R^n$ være et $\beta$-domæne, og lad notationen være
som i definition~\ref{betadomain} og \ref{betamesh}.
$\beta$-netinddelingen $\T_\beta$ siges da at have $n$-firkantede
ele\-menter, hvis ethvert element $\re$ i $\T_\beta$ er
$n$-firkantet. Tilsvarende siges $\T_\beta$ at have $n$-kubiske elementer,
såfremt ethvert element er $n$-kubisk.
\end{definition}
For at påvise eksistensen af net tætheds operatorer vil vi ikke tillade
vilkårlige $\beta$-netinddelinger. Vi vil begrænse os til følgende
\begin{definition}
Ved mængden af $\T^{\ast}$-tilladte $\beta$-netinddelinger med
$n$-kubiske elementer (eng. $\underline{\T}^{\ast}$-\underline{c}ompatible 
$\beta$-meshes with $n$-\underline{c}ubic elements) forkortet $TCC$,
forstås de netinddelinger, der kan konstrueres udfra følgende rekursive regel
\begin{itemize}
  \item $\T^{\ast}$ er et $TCC$.
  \item Hvis $\T$ er et $TCC$, er den $\beta$-netinddeling, der
        fremkommer ved at inddele et vilkårligt element i $\T$ i $2^n$
        kongruente $n$-kuber igen et $TCC$.
  \item En $\beta$-netinddeling, der ikke kan konstrueres udfra
        overnstående, er ikke et $TCC$.
\end{itemize}
\end{definition}
\begin{definition}
Lad $\T^{\ast}$ være en reference $\beta$-netinddeling, $\T$ en
$\T^{\ast}$-tilladt $\beta$-netinddeling med $n$-kubiske elementer, og
lad $\{\re_j\}_{j\in S}$ være en mængde af elementer i $\T$. Hvis
netinddelingen $\T^{'}$, fået udfra $\T$ ved at slå elementerne
$\{\re_j\}_{j\in S}$ sammen til et element $\re_p = \cup_{j\in S}\re_j$, er
en $\T^{\ast}$-tilladt $\beta$-netinddeling med $n$-kubiske elementer,
kaldes $\re_p$ et forældre element, og mængden $\{\re_j\}_{j\in S}$
kaldes et søskende sæt.     
\end{definition}
Vi har nu begreber og notation på plads til at kunne vise eksistensen
af net tætheds operatorer. 

\begin{theorem} \label{operator-net}
Lad $\dom$ være et begrænset $\beta$-domæne i $\R^n$, og lad
$\T^{\ast}$ være en refe\-rence $\beta$-netinddeling af $\dom$. I
algoritmen~\ref{ndoalg} er $\hat{\T}$ en $\T^{\ast}$-tilladt
$\beta$-netinddeling med $n$-kubiske elementer, og $\alpha$ er en
reel, positiv parameter. Endelig er $\D\in D_\dom$ en net tætheds
funktion defineret på $\dom$. Definer nu operatoren $\ndo$ ved
\begin{equation}
  \ndo (\D ) = \{ \T_m \}_{m=1}^\infty ,
\end{equation}
hvor $\{ \T_m \}_{m=1}^\infty$ er den følge $\beta$-netinddelinger,
som man kan opnå ved at anvende algoritme~\ref{ndoalg} med intensiteten $\I$
(den reciprokke af det ``ønskede'' antal elementer) varie\-rende over
alle positive tal, og derefter sorteret så den
tilhørende følge $\{N_{a,m}\}_{m=1}^\infty$ er ikke-aftagende. Da er
$\ndo$ en net tætheds operator. Betegnes med $R_{\ndo} = \ndo(D_\dom)$
billedet af $\ndo$, er den inverse afbildning $(\ndo)^{-1}$ til $\ndo$
givet ved
\begin{equation} \label{defndo}
  (\ndo)^{-1} : \{ \T_m \}_{m=1}^\infty \in R_{\ndo}
  \rightarrow \D \in D_\dom ,
\end{equation}
hvor
\begin{equation} \label{defntc}
  \D (x) = \alpha \lim_{k\rightarrow\infty} 
  \frac{\underline{\I}_{m_k}}{V_n(\re_{m_k}(x))} \quad \text{næsten overalt i $\dom$.}
\end{equation}
I definitionen~\eqref{defntc} af net tætheds funktionen er $\re_m(x)$ det
element i $\T_m =\{\re_{j,m}\}_{m=1}^{N_{a.m}}$, der indeholder $x$.
Bemærk at elementet $\re_m(x)$ kun er næsten
overalt entydig. $V_n(\re )$ er $n$-volumenet af $\re$, og
$\underline{\I}_m$ er infimum over alle de intensiteter $\I$ fra
algoritme~\ref{ndoalg}, der
resulterede i $\beta$-netinddelinger $\T_m$. Endelig er
$\{m_k\}_{k=1}^\infty$ en delfølge af indices valgt således, at
elementet $\re_{m_k}(x)$ fra netinddelingen $\T_{m_k}$ i den efterfølgende netinddeling
$\T_{m_{k+1}}$ er blevet inddelt i $2^n$ kongruente elementer.   
\end{theorem}
\newlength{\len}
\settowidth{\len}{(3) }
\begin{Algoritme}[H] 
\begin{emptylist}
\setlength{\itemsep}{0cm}
\setlength{\parsep}{0cm}
  \item I. Forgrovning \\ 
        \makebox[.25in][r]{} For alle relevante søskende sæt
        $\{\re_j\}_{j\in S}$ i $\hat{\T}$ gøres følgende:
  \item \makebox[.5in][r]{} (1) Udregn $\kappa(\re_p)=(1/\I)\int_{\re_p}
        \D (x)\, dx$, hvor $\re_p=\cup_{j\in S}\re_j$
  \item \makebox[.5in][r]{} (2) Hvis $\kappa(\re_p)\geq\alpha$
        foretages ingenting 
  \item \makebox[.5in][r]{} (3) Hvis $\kappa(\re_p)<\alpha$
         konstrueres en ny $\beta$-netinddeling ved \\
        \makebox[.5in][r]{} \makebox[\len][r]{} 
        at erstatte $\{\re_j\}_{j\in S}$ med $\re_p$ 
  \item II. Forfining \\
        \makebox[.25in][r]{} For alle elementerne $\re$ i
        $\beta$-netinddelingen som blev \\
        \makebox[.25in][r]{} konstrueret i forgrovnings delen gøres følgende:    
  \item \makebox[.5in][r]{} (1) Udregn $\kappa(\re)=(1/\I)\int_{\re}
        \D (x)\, dx$, (hvis det ikke \\
        \makebox[.5in][r]{} \makebox[\len][r]{} allerede er gjort)
  \item \makebox[.5in][r]{} (2) Hvis $\kappa(\re)\leq\alpha$
        forbliver $\re$ uændret 
  \item \makebox[.5in][r]{} (3) Hvis $\kappa(\re)>\alpha$ \\
        \makebox[.5in][r]{} \makebox[\len][r]{} (a) Konstruer en ny
        $\beta$-netinddeling ved at inddele $\re$ i \\
        \makebox[.5in][r]{} \makebox[\len][r]{} \makebox[\len][r]{} 
        $2^n$ kongruente $n$-kuber \\
        \makebox[.5in][r]{} \makebox[\len][r]{} (b) For alle
        underelementerne $\re_s$ af $\re$ gå til II.(1)
\end{emptylist}
\caption{Netinddelings algoritme byggende på net tætheds funktion\label{ndoalg}}
\end{Algoritme}

\begin{remark}
I algoritme~\ref{ndoalg} udføres forgrovning og forfining kun en 
enkelt gang. Dette er valgt for at undgå muligheder for 
uendelige løkker. Forgrovning udføres før forfining for at sikre
en mere effektiv udnyttelse af computer lagerplads. Ved at udføre
forgrovning og forfining i nævnte rækkefølge kan man nemlig 
udnytte den lagerplads, der er blevet frigjort i forbindelse med
forgrovning til at gemme informationer om forfining i. For flere 
detaljer om algoritmen henvises til afsnit~\ref{netalg}, hvor vi 
konstruerer en mere generel algoritme. Den initiale netinddeling
$\hat{\T}$ er bruger bestemt og kan eventuelt vælges som $\T^{\ast}$.
\end{remark}

\begin{proof}
Det er klart, at $A_n^{\ast}(\D)$ er en tilladt ikke-aftagende
netinddeling af $\dom$ for enhver net tætheds funktion $\D$. Dette
følger af den måde, hvorpå netinddelinger i $A_n^{\ast}(\D)$ er 
sorteret samt af, at algoritmen~\ref{ndoalg} er konstueret således, at 
ethvert element på et eller andet tidspunkt bliver inddelt i mindre
elementer når intensiteten aftager. Altså vil det samlede antal 
elementer gå mod uendelig, når intensiteten går mod nul. For at vise 
at det resterende udsagn i sætningen er korrekt, skal vi vise, at 
$\D$ defineret ved~\eqref{defndo} og \eqref{defntc} opfylder 
$A_n^{\ast}(\D) = \{\T_m\}_{m=1}^\infty$, hvor $\T_m$ er netinddelingen
fået fra algoritme~\ref{ndoalg}. Da $\{\T_m\}_{m=1}^\infty$ antages at 
tilhøre billedet af $A_n^{\ast}$, findes der en net tætheds funktion
$\tilde{\D}$, så $A_n^{\ast}(\tilde{\D})=\{\T_m\}_{m=1}^\infty$. Lad
$S_0$ betegne den mængde, hvor $\tilde{\D}$ er diskontinuert eller 
har singulariteter. Mængden $S_0$ har mål $0$. Bemærk at mængden 
$S\equiv S_0 \cup ( \cup_{m=1}^\infty \cup_{j=1}^{N_{a,m}} \partial
\Box_{j,m})$ ligeledes har mål $0$. Sæt nu
$a_m=\int_{\Box_m(x)} \tilde{\D}\, dx$, og lad $\I_m$ være et 
vilkårligt tal. Ved at indsætte $\I_m$ for $\I$ i algoritmen~\ref{ndoalg}
fås en netinddeling $\{\T_m\}_{m=1}^\infty$. Ifølge algorimen vil en
forfining af $\Box_m(x)$ kun ske såfremt $a_m(x)/\I_m > \alpha$
Men ifølge konstruktionen af algoritmen og definitionen af $a_m$ vil 
$a_m(x)/\I_m \leq \alpha$. Fra definitionen af indeks delfølgen
$\{ m_k \}_{k=1}^\infty$ følger det nu, at $a_{m_k}/\underline{\I}_{m_k}
= \alpha$ for alle positive heltal $k$. Vi har nu
\begin{align} \label{helpequ}
  \tilde{\D}(x) & = \lim_{m\rightarrow\infty}
    \frac{a_{m_k}(x)}{V_n(\Box_{m_k}(x))} \\
  & = \frac{a_{m_k}(x)}{\underline{\I}_{m_k}} 
    \lim_{m\rightarrow\infty} 
    \frac{\underline{\I}_{m_k}}{V_n(\Box_{m_k}(x))} \notag \\
  & = \alpha \lim_{m\rightarrow\infty} \frac{\underline{\I}_{m_k}}{V_n(\Box_{m_k}(x))} 
    \quad \text{for $x\in\overline{\dom}\setminus S$.}\notag
\end{align}
Da $S$ har mål $0$ følger det af~\eqref{helpequ}, at $\D$ givet ved~\eqref{defntc} 
er den rigtige net tætheds funktion, da den er identiske med $\tilde{\D}$
næsten overalt.
\end{proof}

\section{Netgenerering i to dimensioner}
Vi skal fremover antage, at domænet er en delmængde af $\R^2$, samt at
dette kan afbildes bijektivt på et passende antal enheds kvadrater
$[0,1]^2$. For at forenkle notationen skal vi som hovedregel kun betragte et enkelt
generisk enheds kvadrat. Elementerne i de endelige element net
$\T_N=\{\re_j\}_{j=1}^N$, som vi skal arbejde med består udelukkende af
rektangulære elementer, se figur~\ref{meshexample} for et eksempel og
notation. Vi tillader altså her mere generelle elementer end i
sætning~\ref{operator-net} fra forrige afsnit.
\begin{figure}[hbt]
\begin{center}
\setlength{\unitlength}{1cm}
\begin{picture}(12,6)(0,0)
\put(0,0){\begin{picture}(6,6)(0,0)
\put(1,2){\line(1,0){4}}
\put(1,2){\line(0,1){4}}
\put(5,6){\line(-1,0){4}}
\put(5,6){\line(0,-1){4}}
\put(2,3){\line(1,0){1}}
\put(1,4){\line(1,0){2}}
\put(1,5){\line(1,0){2}}
\put(2,2){\line(0,1){4}}
\put(3,2){\line(0,1){4}}
\put(4.5,2){\line(0,1){4}}
\put(4,2){\line(0,1){4}}
\put(3,4){\line(1,0){1}}
\put(1,1.5){Net $\T_N = \{\re_j\}_{j=1}^N$}
\put(1,1){på $[0,1]^2$, $(N=11)$} 
\end{picture}}
\put(6,0){\begin{picture}(6,6)(0,0)
\put(1,3){\line(1,0){4}}
\put(1,3){\line(0,1){3}}
\put(5,6){\line(-1,0){4}}
\put(5,6){\line(0,-1){3}}
\put(3,2.5){\makebox(0,0){$h_{1,j}$}}
\put(5.1,4.5){\makebox(0,0)[l]{$h_{2,j}$}}
\put(1,1.5){Forstørrelse af et}
\put(1,1){element $\re_j$}
\put(3,4.5){\makebox(0,0){$\re_j$}} 
\end{picture}}
\end{picture}
\end{center}
\caption{Net og element notation}
\label{meshexample}
\end{figure}
Netinddelings algoritmen, som vi skal betragte, kræver, at vi har
en a priori fejlligning på formen
\begin{equation} \label{fejlligning}
  \norm{e}^q_{\re_j} = \phi_1^q h_{1,j}^{\gamma +1} h_{2,j} +
  \phi_2^q h_{1,j} h_{2,j}^{\gamma +1} + \hot \quad 
  \text{for $j=1,\ldots,N\ $.}
\end{equation}
I ligningen~\eqref{fejlligning} er $\re_j$ de rektangulære elementer,
som det endelige element net over det afbildede domæne består af.
Sidelængderne af $\re_j$
betegnes med $h_{1,j}$ og $h_{2,j}$ i $x$-retningen hhv.
$y$-retningen. Vi skriver $\hot$ for højere ordens led i $h_{1,j}$ og
$h_{2,j}$. Funktionerne $\phi_1$ og $\phi_2$ er kendte men
ikke-beregnelige funktioner, der afhænger af den funktion, vi
approksimerer, men er uafhængige af netinddelingen. Kendes den
funktion, som vi forsøger at approksimere, er $\phi_1$ og $\phi_2$
beregnelige. Endelig er $q$ et positivt heltal, og $\gamma$ er en reel
parameter afhængig af polynomiums graden $p$ (på de polynomier som vi
approksimerer med) og normen $\norm{\cdot}$.
Det er underforstået, at den globale fejl opfylder følgende
summations princip
\begin{equation} \label{sumprincip}
  \norm{e}^q_\dom = \sum_{j=1}^N \norm{e}^q_{\re_j}.
\end{equation}
I special tilfældet hvor alle elementer er kvadratiske, reduceres
fejlligningen~\eqref{fejlligning} til $\norm{e}^q_{\re_j} = 2\phi^q
h_j^{\gamma +2} + \hot $ for $j=1,\ldots,N$, hvor
$2\phi^q=\phi_1^q+\phi_2^q$.

Vi så i kapitel~\ref{interpolation}, at det er muligt at udlede en a priori
fejlligning opfyldende oven\-stå\-en\-de krav. 

\subsection{Konstruktion af næsten optimale net størrelses funktioner
for netinddelinger med rektangulære elementer}
Betragt nu en følge af net $\{\T_N\}_{N=1}^\infty$, og lad
$\T_N=\{\re_j\}_{j=1}^N$ være et fast generisk net i følgen svarende
til et fast valgt $N$. For dette net kan vi opfatte længden af siderne
$h_{1,j}$ og $h_{2,j}$ i elementet $\re_j$ som værdien af 
stykkevis konstante, ikke-negative (net størrelses) funktioner $h_1$ og
$h_2$. Det er oplagt, at disse opfylder
\begin{equation} \label{trivial}
  \int_\dom \frac{1}{Nh_1 h_2}\, dx\, dy = 
  \frac{1}{N} \sum_{j=1}^N \int_{\re_j} \frac{1}{h_1 h_2}\, dx\, dy = 1.  
\end{equation}
Fjernes betingelsen ``stykkevis konstant'' fås følgende generelle
definition af net stør\-rel\-ses funktioner.
\begin{definition}
For et givet endeligt element net $\T_N=\{\re_j\}_{j=1}^N$ defineres
net størrelse funktioner $h_1$ og $h_2$ som værende to
ikke-negative, næsten overalt glatte funktioner
opfyldende~\eqref{trivial} og 
\begin{equation} \label{def-mfc}
  \int_{\re_j} \frac{1}{h_1}\, dx\, dy = h_{2,j}
  \quad \text{og} \quad
  \int_{\re_j} \frac{1}{h_2}\, dx\, dy = h_{1,j}
  \quad \text{for $j=1,\ldots,N$.}
\end{equation}
\end{definition}

\begin{remark}
Definitionen af net størrelses funktioner kan umiddelbart
ge\-ne\-ra\-li\-se\-res til højere dimensioner.
\end{remark}

\begin{remark}
Ifølge definition~\ref{density} er den næsten overalt glatte funktion
$\D=\frac{1}{Nh_1h_2}$ en net tætheds funktion. I tilfældet med lutter
kvadratiske elementer er $h_1 \equiv h_2 \equiv h$, hvorfor
$\D=\frac{1}{Nh^2}$.  
\end{remark}

\begin{remark}
Kravene i ligning~\eqref{def-mfc} implicerer, at $h_i(x,y)$ op til
højere ordens led er identisk med $h_{i,j}$ for ethvert punkt
$(x,y)\in\re_j$ (og $i=1,2$, $j=1,\ldots,N$). 
\end{remark}

\begin{remark}
Har vi givet to net størrelses funktioner $h_1$ og $h_2$ beskrivende
en eller flere netinddelinger, kan vi ved brug af~\eqref{def-mfc}
rekonstruere disse netfølger. 
\end{remark}

I praktiske beregningssituationer forventer man ikke, at kunne
konstruere net opfyldende~\eqref{def-mfc}. Man vil
derfor normalt tillade approksimationer af typen 
\begin{equation} \label{approxkK}
  k \leq \frac{1}{h_{2,j}} \int_{\re_j} \frac{1}{h_1}\, dx\,dy \leq K 
  \quad \text{og} \quad
  k \leq \frac{1}{h_{1,j}} \int_{\re_j} \frac{1}{h_2}\, dx\,dy \leq K,
\end{equation}
for $j=1,\ldots,N$ og passende konstanter $k$ og $K$ med $k\leq K$.

Vi er interesseret i at konstruere net størrelses funktioner
for asymptotisk optimale net følger
$\{\hat{\T}_N\}_{N=1}^\infty$, hvor $\hat{\T}_N$ er det net med den
mindst mulige fejl $\norm{\hat{e}}_\dom$ blandt alle net med $N$ elementer. Fejlen
er her beskrevet ved $\norm{\hat{e}}_\dom$ som fås fra \eqref{fejlligning} ved
at udelade højere ordens led (bemærk at dette intet ændrer
asymptotisk). Vi skal nu forklare, hvorledes man kan
konstruere asymptotiske optimale net størrelse funktioner
$\hat{h}_1$ og $\hat{h}_2$ (og dermed en asymptotisk optimal net tætheds
funktion $\D$) hørende til $\hat{\T}_N$. Hertil skal vi benytte
ligningerne~\eqref{trivial} og \eqref{fejlligning}, hvor vi i
ligning~\eqref{fejlligning}, skal erstatte $h_{i,j}$ med $h_i$ i hvert
element $\re_j$. Vi skal desuden se bort fra højere ordens led.

Ved som sagt at erstatte $h_{i,j}$ med $h_i$ i \eqref{fejlligning} og
summere over $j=1,\ldots,N$ fås
\begin{equation}
  \norm{\hat{e}}^q_\dom = \int_\dom (\phi_1^q h_1^{\gamma} + 
  \phi_2^q h_2^{\gamma})\, dx\, dy.
\end{equation}
Bemærk at vi benytter skrivemåden $\hat{e}$ for at indikere, at vi ser
bort fra højere ordens led. Arbejder vi med kvadratiske elementer fås
istedet $\norm{\hat{e}}^q_\dom = \int_\dom 2\phi^q h^\gamma\, dx\, dy$.

For at bestemme $\hat{h}_1$ og $\hat{h}_2$ skal vi minimere
$\norm{\hat{e}}^q_\dom$ for fastholdt $N$, dvs vi skal løse følgende
minimeringsproblem:
\begin{align} \label{minimum}
  &\text{Bestem de net størrelses funktioner $\hat{h}_1$ og 
    $\hat{h}_2$} \\
  &\text{som minimerer udtrykket $\int_\dom (\phi_1^q h_1^{\gamma} + 
    \phi_2^q h_2^{\gamma})\, dx\, dy$}  \notag \\
  &\text{under bibetingelse $\int_\dom \frac{1}{Nh_1h_2}\, dx\, dy=1$,
    hvor $N$ er fast.} \notag
\end{align}
Dette minimeringsproblem kan løses fx ved hjælp af Lagrange optimering
som følger: Definer Lagrange funktionen $L$ som
\begin{equation}
  L(h_1,h_2,\lambda) = \int_\dom (\phi_1^q h_1^\gamma + 
  \phi_2^q h_2^\gamma )\, dx\, dy + \lambda \Bigl(
  \int_\dom \frac{1}{Nh_1h_2}\, dx\, dy -1 \Bigr). 
\end{equation}
De afledede af $L$ med hensyn til $h_1$ og $h_2$ er
\begin{align}
  \delta L(h_1,h_2,\lambda) &= 
    \int_\dom \gamma\bigl( \phi_1^q h_1^{\gamma -1}\delta h_1 + 
  \phi_2^q h_2^{\gamma -1} \delta h_2 \bigr)\, dx\, dy \\
  &\phantom{=\int} -\frac{\lambda}{N} \int_\dom \Bigl(
    \frac{1}{h_1^2 h_2} \delta h_1 + \frac{1}{h_1 h_2^2} \delta h_2
    \Bigr)\, dx\, dy . \notag  
\end{align}
Sættes $\delta L$ til $0$ for vilkårlige variationer $\delta h_1$ og
$\delta h_2$ fås følgende optimalitets betingelser 
\begin{gather}
  \phi_1^q \hat{h}_1^{\gamma +1} \hat{h}_2 = 
    \frac{\lambda}{\gamma N} = C, \label{opc1} \\
  \phi_2^q \hat{h}_1 \hat{h}_2^{\gamma +1} = 
    \frac{\lambda}{\gamma N} = C. \label{opc2}
\end{gather}
Hvor det sidste lighedstegn i begge ovenstående ligninger definerer
den globale konstant $C$.

Ved nu at anvende ligningerne~\eqref{fejlligning}, \eqref{opc1} og
\eqref{opc2} samt det faktum, at $h_i$ og $h_{i,j}$ op til højere
ordens led er identiske i
$\re_j$ fås 
\begin{align} \label{error-eq-dis}
  \norm{\hat{e}}_{\re_j}^q &= \phi_1^q \hat{h}_1^{\gamma +1} \hat{h}_2
    + \phi_2^q \hat{h}_1 \hat{h}_2^{\gamma +1} + \hot \\
  &= 2C + \hot \quad \text{for $j=1,\ldots,N$.} \notag
\end{align}
Heraf ses, at den asymptotisk optimale fejl $\norm{\hat{e}}_\re$ er
uafhængig af elementet $\re$, dvs. fejlen er ækvidistribueret. 

Værdien af konstanten $C$ kan bestemmes fra bibetingelsen $\int_\dom
\frac{1}{Nh_1h_2}\, dx\, dy=1$ på følgende måde: multiplicer
ligningerne~\eqref{opc1} og \eqref{opc2}, opløft til $1/(\gamma +2)$'te
potens, isoler $\frac{1}{N\hat{h}_1 \hat{h}_2}$ på den ene side, integrer over
$\dom$ og sæt resultatet lig med en dvs.
\begin{align}
  &\phantom{\Downarrow} (\phi_1 \phi_2)^q \hat{h}_1^{\gamma +2} 
    \hat{h}_2^{\gamma +2} = C^2 \notag \\    
  &\Downarrow \notag \\
  &\phantom{\Downarrow}  (\phi_1 \phi_2)^{\frac{q}{\gamma +2}}
    \hat{h}_1 \hat{h}_2 = C^{\frac{2}{\gamma +2}} \notag \\   
  &\Downarrow \notag \\
  &\phantom{\Downarrow}  \frac{1}{N} C^{\frac{\gamma +2}{2}} \int_\dom
    (\phi_1 \phi_2)^{\frac{q}{\gamma +2}}\, dx\, dy = \int_\dom
    \frac{1}{N\hat{h}_1 \hat{h}_2}\, dx\, dy  \notag \\   
  &\Downarrow \notag \\
  &\phantom{\Downarrow}  \frac{1}{N} C^{\frac{\gamma +2}{2}} \int_\dom
    (\phi_1 \phi_2)^{\frac{q}{\gamma +2}}\, dx\, dy = 1 \notag \\
  &\Downarrow \notag \\
  &\phantom{\Downarrow} C = \Bigl( \frac{1}{N}\int_\dom (\phi_1\phi_2)^{q/(\gamma+2)} 
    \, dx\, dy \Bigr)^{\frac{\gamma +2}{2}}. \notag     
\end{align}
Da konstanten $C$ nu er kendt, kan vi ved hjælp at \eqref{opc1} og
\eqref{opc2} bestemme $\hat{h}_1$ og $\hat{h}_2$. Isoleres $\hat{h}_2$
i \eqref{opc1}, og indsættes resultatet så i \eqref{opc2}, kan vi
bestemme $\hat{h}_1$ til
\begin{align} \label{opt-h1}
  \hat{h}_1 &= \bigl( C^\gamma \phi_1^{-q(\gamma +1)} \phi_2^q 
    \bigr)^{\frac{1}{\gamma^2 +2\gamma}} \\
  &= \frac{\overline{C}}{\sqrt{N}} \bigl( \phi_1^{-q(\gamma +1)} \phi_2^q 
    \bigr)^{\frac{1}{\gamma^2 +2\gamma}}. \notag
\end{align}
Her er konstanten $\overline{C}$ uafhængig af nettet og givet ved
\begin{equation} \label{valueofC}
  \overline{C} = \Bigl( \int_\dom (\phi_1 \phi_2)^{\frac{q}
  {\gamma +2}}\, dx\, dy \Bigr)^{\frac{1}{2}}.
\end{equation}
På tilsvarende vis bestemmes $\hat{h}_2$
\begin{align} \label{opt-h2}
  \hat{h}_2 &= \bigl( C^\gamma \phi_1^q \phi_2^{-q(\gamma +1)} 
    \bigr)^{\frac{1}{\gamma^2 +2\gamma}} \\
  &= \frac{\overline{C}}{\sqrt{N}} ( \phi_1^q \phi_2^{-q(\gamma +1)}
     )^{\frac{1}{\gamma^2 +2\gamma}}. \notag
\end{align}
Er elementerne kvadratiske fås
$\hat{h}=\frac{\overline{C}}{\sqrt{N}}\phi^{-q/(\gamma +2)}$ med
$\overline{C}=\bigl( \int_\dom \phi^{2q/(\gamma +2)}\, dx\, dy
\bigr)^{1/2}$.

I begge situationer er minimums værdien for \eqref{minimum} givet ved
\begin{equation} \label{min-value}
  \int_\dom \bigl( \phi_1^q \hat{h}_1^\gamma +
  \phi_2^q \hat{h}_2^\gamma \bigr)\, dx\, dy = 
  \frac{2\overline{C}^{\gamma +2}}{N^{\frac{\gamma}{2}}}.
\end{equation}

Endelig bemærkes at $\hat{h}_1\sqrt{N}$ og $\hat{h}_2\sqrt{N}$ er
uafhængige af netinddelingen. Det betyder, at man alene ved at gemme
de asymptotisk optimale net størrelses funktioner kan rekonstruere hele netfølgen.

\subsection{Bestemmelse af det næsten optimale antal af elementer for
rektangulære netinddelinger} \label{noae}
I forrige afsnit så vi, hvordan man kan bestemme asymptotisk optimale
net størrelses funktioner til beskrivelse af den
asymptotisk optimale netinddeling. I dette afsnit skal vi se, hvordan
vi bestemmer antallet af elementer i dette. I
praktiske situationer er vi interesseret i tilfældet, hvor brugeren
kræver en løsning med en fejl mindre end en vis given tolerance
$\tau$, dvs. brugeren kræver $\norm{e}_\dom\leq\tau$. Ved hjælp af
$\hat{h}_1$ og $\hat{h}_2$ kan vi bestemme det minimale antal
elementer $\hat{N}$, der kræves, for at den asymptotisk optimale
netinddeling opfylder de specificerede krav til fejlen. Det kan gøres
ved at løse følgende minimerings problem:  
\begin{align} \label{cononN}
  &\text{Bestem det mindste tal $\hat{N}$ som opfylder} \\
  &\int_\dom \bigl( \phi_1^q \hat{h}_1^\gamma + \phi_2^q \hat{h}_2^\gamma 
    \bigr)\, dx\, dy \leq \tau^q. \notag
\end{align}
Er elementerne kvadratiske lyder minimerings problemet istedet: Bestem
det mindste $\hat{N}$ så $\int_\dom 2\phi^q\hat{h}^\gamma
\, dx\, dy \leq\tau^q$.
Løsningen til dette problem findes let ved brug af
formel~\eqref{min-value} og er givet ved 
\begin{equation} \label{valueofN}
  \hat{N} = \Biggl\lceil \Bigl( \frac{2\overline{C}^{\gamma +2}}{\tau^q}
  \Bigr)^{\frac{2}{\gamma}} \Biggr\rceil .   
\end{equation}
For kvadratiske elementer fås samme formel, men her er 
$\overline{C}=( \int_\dom \phi^{2q/(\gamma +2)}\, dx\, dy )^{1/2}$.

\subsection{Netinddelings algoritmen} \label{netalg}
Vi er nu i stand til at udlede algoritmen til konstruktion af den
optimale netinddeling. Dette vil blive gjort udfra de informationer,
som vi har til rådighed fra de asymptotisk optimale net størrelses
funktioner samt det asymptotisk optimale antal af elementer. Af
notationsmæssige hensyn vil vi kun betragte netinddelinger af et enkelt
generisk kvadrat i rektangulære elementer. For at få en simpel
netinddelings algoritme vil vi ikke tillade vilkårlige rektangler.
Dette leder til følgende definition
\begin{definition}
Mængden af tilladte netinddelinger (eng. the class of admissable
meshes) defineres som følgende:
\begin{itemize}
  \item Enheds kvadratet er en tilladt netinddeling.
  \item Hvis $\T^{'}$ er konstrueret udfra en tilladt netinddeling
        $\T$ ved at opdele et element i $\T$ i to kongruente 
        rektangler, er $\T^{'}$ en tilladt netinddeling. 
  \item Ingen andre netinddelinger er tilladte.
\end{itemize}
\end{definition}
\begin{remark}
I artiklen \cite{hugger-net} er algoritmen testet på en række eksempler. I
disse numeriske test har man arbejdet med endnu en betingelse i
definitionen af tilladte netinddelinger. Der henvises til \cite{hugger-net} for
detaljer. Dette ekstra krav har ingen betydning for de teoretiske
resultater, som vi skal vise i afsnit~\ref{teorires} og er udelukkende
medtaget af implementationsmæssige årsager. 
\end{remark}
I diskussionen fremover vil det være hensigtsmæssigt at have fastlagt
en datastruktur for tilladte netinddelinger. I figur~\ref{adm-net}
findes et eksempel på en tilladt netinddeling. Den tilhørende
datastruktur kan repræsenteres ved et træ, se figur~\ref{datastructur}
\begin{figure}[p]
\begin{center}
\setlength{\unitlength}{1cm}
\begin{picture}(8,8)(0,0)
\put(0,0){\line(1,0){8}}
\put(0,0){\line(0,1){8}}
\put(8,8){\line(-1,0){8}}
\put(8,8){\line(0,-1){8}}
\put(0.5,0){\line(0,1){8}}
\put(1,0){\line(0,1){8}}
\put(2,0){\line(0,1){8}}
\put(4,0){\line(0,1){8}}
\put(0,4){\line(1,0){0.5}}
\put(4,4){\line(1,0){4}}
\put(4,4.5){\line(1,0){1}}
\put(4,5){\line(1,0){2}}
\put(4,6){\line(1,0){4}}
\put(4.5,4){\line(0,1){1}}
\put(5,4){\line(0,1){2}}
\put(6,4){\line(0,1){2}}
\put(6,2){\line(1,0){2}}
\put(6,1){\line(1,0){2}}
\put(6,0.5){\line(1,0){2}}
\put(6,0){\line(0,1){4}}
\put(6.5,0){\line(0,1){1}}
\put(7,0){\line(0,1){1}}
\put(7.5,0){\line(0,1){1}}
\put(3,4){\makebox(0,0){1}}
\put(1.5,4){\makebox(0,0){2}}
\put(0.75,4){\makebox(0,0){3}}
\put(0.25,6){\makebox(0,0){4}}
\put(0.25,2){\makebox(0,0){5}}
\put(5,2){\makebox(0,0){11}}
\put(7,3){\makebox(0,0){12}}
\put(7,1.5){\makebox(0,0){13}}
\put(6.25,0.75){\makebox(0,0){18}}
\put(6.75,0.75){\makebox(0,0){19}}
\put(6.25,0,25){\makebox(0,0){20}}
\put(6.75,0.25){\makebox(0,0){21}}
\put(7.25,0.75){\makebox(0,0){22}}
\put(7.25,0.25){\makebox(0,0){23}}
\put(7.75,0.75){\makebox(0,0){24}}
\put(7.75,0.25){\makebox(0,0){25}}
\put(6,7){\makebox(0,0){6}}
\put(4.5,5.5){\makebox(0,0){8}}
\put(5.5,5.5){\makebox(0,0){9}}
\put(4.25,4.75){\makebox(0,0){14}}
\put(4.25,4.25){\makebox(0,0){16}}
\put(4.75,4.75){\makebox(0,0){15}}
\put(4.75,4.25){\makebox(0,0){17}}
\put(7,5){\makebox(0,0){7}}
\put(5.5,4.5){\makebox(0,0){10}}
\end{picture}
\end{center}
\caption{Eksempel på tilladt netinddeling\label{adm-net}}
\end{figure}
\begin{figure}[p]
\begin{center}
\setlength{\GapDepth}{1cm}
\setlength{\GapWidth}{0.25cm}
\begin{bundle}{}
  \chunk[$x$]{\begin{bundle}{} 
    \chunk[$x$]{\begin{bundle}{} 
      \chunk[$x$]{\begin{bundle}{} 
        \chunk[$x$]{\begin{bundle}{} 
                      \chunk[$y$]{4}      
                      \chunk[$y$]{5}
                    \end{bundle}}  
                    \chunk[$x$]{3}
                  \end{bundle}}    
                  \chunk[$x$]{2}
                \end{bundle}}
                \chunk[$x$]{1}
              \end{bundle}}
  \chunk[$x$]{\begin{bundle}{}
    \chunk[$y$]{\begin{bundle}{}
      \chunk[$y$]{6}
      \chunk[$y$]{\begin{bundle}{}
        \chunk[$x$]{7}
        \chunk[$x$]{{\begin{bundle}{}
          \chunk[$xy$]{8}
          \chunk{9}
          \chunk{10}
          \chunk[$xy$]{\begin{bundle}{} 
            \chunk[$xy$]{14}
            \chunk{15}
            \chunk{16}
            \chunk[$xy$]{17}
                       \end{bundle}} 
                     \end{bundle}}}
                  \end{bundle}}
                \end{bundle}}
    \chunk[$y$]{\begin{bundle}{}
      \chunk[$y$]{11}
      \chunk[$y$]{\begin{bundle}{}
        \chunk[$x$]{12}
        \chunk[$x$]{\begin{bundle}{}
          \chunk[$y$]{13}
          \chunk[$y$]{\begin{bundle}{} 
            \chunk[$xy$]{\begin{bundle}{}
              \chunk[$xy$]{18}
              \chunk{19}
              \chunk{20}
              \chunk[$xy$]{21}
                         \end{bundle}}
            \chunk[$xy$]{\begin{bundle}{}
              \chunk[$xy$]{22}
              \chunk{23}
              \chunk{24}
              \chunk[$xy$]{25}
                         \end{bundle}}
                       \end{bundle}} 
                     \end{bundle}}
                  \end{bundle}}
                \end{bundle}}
              \end{bundle}} 
\end{bundle}
\end{center}
\caption{Datastruktur for netinddelingen i figur 5.3\label{datastructur}}
\end{figure}

Et vilkårligt element kan forfines på tre forskellige måder. Disse vil
blive betegnes med $x$, $y$ og $xy$ inddeling.
Figur~\ref{pos-subdivisions} indeholder en forklaring af notationen. I
forbindelse med forgrovning vil vi kun betragte en bestemt type af
søskende sæt med 2 eller 4 elementer. For disse 2 eller 4 elementer
skal der gælde, at de udgør samtlige blade i en given gren i træet, og
at denne gren ikke har nogle undergrene. Der findes tre muligheder
for søskende sæt af ovennævnte type betegnet med hhv. type $x$, $y$ og
$xy$, se figur~\ref{pos-subdivisions}(a). For et konkret eksempel
henvises til figur~\ref{adm-net}. Her udgør elementerne 4 og 5 et
søskende sæt af type $y$, mens elementerne 8 og 9 ikke er et søskende
sæt i ovenstående forstand. Den valgte datastruktur er
ikke entydig, idet identiske netinddelinger kan have forskellig træ
repræsentation afhængig af, hvordan de er opnået. Der henvises til
\cite{hugger-net} for et eksempel. Da diskussionen her er teoretisk har det
ingen  betydning. 
\begin{figure}[htb]
\begin{center}
\setlength{\unitlength}{1cm}
\begin{picture}(12,5)(0,0)
\put(0.5,3.75){\makebox(0,0){(a)}}
\put(1.5,3){\line(1,0){2}}
\put(4.5,3){\line(1,0){2}}
\put(7,3){\line(1,0){2}}
\put(9.5,3){\line(1,0){2}}
\put(1.5,4.5){\line(1,0){2}}
\put(4.5,4.5){\line(1,0){2}}
\put(7,4.5){\line(1,0){2}}
\put(9.5,4.5){\line(1,0){2}}
\put(1.5,3){\line(0,1){1.5}}
\put(3.5,3){\line(0,1){1.5}}
\put(4.5,3){\line(0,1){1.5}}
\put(6.5,3){\line(0,1){1.5}}
\put(7,3){\line(0,1){1.5}}
\put(9,3){\line(0,1){1.5}}
\put(9.5,3){\line(0,1){1.5}}
\put(11.5,3){\line(0,1){1.5}}
\put(3.6,3.75){\vector(1,0){0.8}}
\put(2.5,3.75){\makebox(0,0){$\re$}}
\put(5,3.75){\makebox(0,0){$\re_1$}}
\put(6,3.75){\makebox(0,0){$\re_2$}}
\put(6.75,3.75){\makebox(0,0){,}}
\put(8,3.375){\makebox(0,0){$\re_2$}}
\put(8,4.125){\makebox(0,0){$\re_1$}}
\put(9.25,3.75){\makebox(0,0){,}}
\put(10,3.375){\makebox(0,0){$\re_3$}}
\put(10,4.125){\makebox(0,0){$\re_1$}}
\put(11,3.375){\makebox(0,0){$\re_4$}}
\put(11,4.125){\makebox(0,0){$\re_2$}}
\put(5.5,3){\line(0,1){1.5}}
\put(10.5,3){\line(0,1){1.5}}
\put(7,3.75){\line(1,0){2}}
\put(9.5,3.75){\line(1,0){2}}
\put(0.5,1.25){\makebox(0,0){(b)}}
\put(2.5,2.5){\circle*{0.1}}
\put(2.5,2){\makebox(0,0){$\re$}}
\put(3.6,1.75){\vector(1,0){0.8}}
\put(5.5,2.5){\circle*{0.1}}
\put(8,2.5){\circle*{0.1}}
\put(10.5,2.5){\circle*{0.1}}
\put(4.5,1){\circle*{0.1}}
\put(6.5,1){\circle*{0.1}}
\put(7,1){\circle*{0.1}}
\put(9,1){\circle*{0.1}}
\put(9.5,1){\circle*{0.1}}
\put(11.5,1){\circle*{0.1}}
\put(4.5,1){\line(2,3){1}}
\put(6.5,1){\line(-2,3){1}}
\put(7,1){\line(2,3){1}}
\put(9,1){\line(-2,3){1}}
\put(9.5,1){\line(2,3){1}}
\put(11.5,1){\line(-2,3){1}}
\put(10.2,1){\line(1,5){0.3}}
\put(10.8,1){\line(-1,5){0.3}}
\put(10.2,1){\circle*{0.1}}
\put(10.8,1){\circle*{0.1}}
\put(4.5,0.5){\makebox(0,0){$\re_1$}}
\put(6.5,0.5){\makebox(0,0){$\re_2$}}
\put(7,0.5){\makebox(0,0){$\re_1$}}
\put(9,0.5){\makebox(0,0){$\re_2$}}
\put(9.5,0.5){\makebox(0,0){$\re_1$}}
\put(10.2,0.5){\makebox(0,0){$\re_2$}}
\put(10.8,0.5){\makebox(0,0){$\re_3$}}
\put(11.5,0.5){\makebox(0,0){$\re_4$}}
\put(4.5,4.5){\makebox(2,0.5){type $x$}}
\put(7,4.5){\makebox(2,0.5){type $y$}}
\put(9.5,4.5){\makebox(2,0.5){type $xy$}}
\put(6.75,1.75){\makebox(0,0){,}}
\put(9.25,1.75){\makebox(0,0){,}}
\put(4.75,1.75){\makebox(0,0){$x$}}
\put(6.25,1.75){\makebox(0,0){$x$}}
\put(7.25,1.75){\makebox(0,0){$y$}}
\put(8.75,1.75){\makebox(0,0){$y$}}
\put(9.75,1.75){\makebox(0,0){$xy$}}
\put(11.25,1.75){\makebox(0,0){$xy$}}
\end{picture}
\end{center}
\caption{(a) Mulige forfininger af et element og (b) den 
resulterende data struktur\label{pos-subdivisions}}
\end{figure}

I overensstemmelse med ligning~\eqref{def-mfc} vil netinddelings algoritmen
blive baseret på funktionerne
\begin{align}
  \kappa_1(\re) &= \frac{h_1}{\hat{h}_1(\overline{x},\overline{y})} =
    \frac{1}{h_2}\int_\re \frac{1}{\hat{h}_1}\, dx\, dy + \hot , \label{kappa1} \\
  \kappa_2(\re) &= \frac{h_2}{\hat{h}_2(\overline{x},\overline{y})} =
    \frac{1}{h_1}\int_\re \frac{1}{\hat{h}_2}\, dx\, dy + \hot , \label{kappa2}
\end{align}
hvor $\re$ er et vilkårligt element med sidelængde $h_1$ i
$x$-retningen og sidelængde $h_2$ i $y$-retningen. De asymptotisk
optimale net størrelses funktioner betegnes som sædvanligt
med $\hat{h}_1$ og $\hat{h}_2$. Punktet $(\overline{x},\overline{y})$
er et vilkårligt punkt i $\re$. I de numeriske test i \cite{hugger-net} er
$(\overline{x},\overline{y})$ valgt som centrum af $\re$.

I netinddelings algoritmen har vi brug for nedenstående simple
observation. For et søskende sæt af type $xy$ (med notation som i
figur~\ref{siblingset}) gælder følgende relation
\begin{align}
  \kappa_1(\re_1\cup\re_3) &= \frac{1}{2h_y} \int_{\re_1\cup\re_3}
    \frac{1}{\hat{h}_1}\, dx\, dy + \hot \\
  &= \frac{1}{2}\Bigl( \frac{1}{h_y}\int_{\re_1} \frac{1}{\hat{h}_1}\, dx\, dy +
    \frac{1}{h_y}\int_{\re_3} \frac{1}{\hat{h}_1}\, dx\, dy \Bigr) + \hot \notag \\
  &= \frac{1}{2} \Bigl( \kappa_1(\re_1) + \kappa_1(\re_3) \Bigr) + \hot \notag
\end{align}
Tilsvarende ses nedenstående relationer at være opfyldt
\begin{align} \label{kapparelations}
  \kappa_1(\re_1\cup\re_2) &= \kappa_1(\re_1)+\kappa_1(\re_2)+\hot \\ 
  \kappa_1(\re_3\cup\re_4) &= \kappa_1(\re_3)+\kappa_1(\re_4)+\hot \notag \\ 
  \kappa_2(\re_1\cup\re_3) &= \kappa_2(\re_1)+\kappa_2(\re_3)+\hot \notag \\
  \kappa_2(\re_2\cup\re_4) &= \kappa_2(\re_2)+\kappa_2(\re_4)+\hot \notag \\
  \kappa_1(\re_1\cup\re_2\cup\re_3\cup\re_4) &= 
    \half \bigl(\kappa_1(\re_1)+\kappa_1(\re_2)+
    \kappa_1(\re_3)+\kappa_1(\re_4)\bigr) + \hot \notag \\
  \kappa_1(\re_1\cup\re_3) &= \half \bigl(\kappa_1(\re_1)+\kappa_1(\re_3)
    \bigr) +\hot \notag \\
  \kappa_1(\re_2\cup\re_4) &= \half \bigl(\kappa_1(\re_2)+\kappa_1(\re_4)
    \bigr) +\hot \notag \\     
  \kappa_2(\re_1\cup\re_2\cup\re_3\cup\re_4) &= 
    \half \bigl(\kappa_2(\re_1)+\kappa_2(\re_2)+
    \kappa_2(\re_3)+\kappa_2(\re_4)\bigr) + \hot \notag \\  
  \kappa_2(\re_1\cup\re_2) &= \half \bigl(\kappa_2(\re_1)+\kappa_2(\re_2) 
    \bigr) +\hot \notag \\
  \kappa_2(\re_3\cup\re_4) &= \half \bigl(\kappa_2(\re_3)+\kappa_2(\re_4) 
    \bigr) +\hot \notag
\end{align}
\begin{figure}[htb]
\begin{center}
\setlength{\unitlength}{1cm}
\begin{picture}(6,4)(0,0)
\put(1,1){\line(1,0){3}}
\put(1,2){\line(1,0){3}}
\put(1,3){\line(1,0){3}}
\put(1,1){\line(0,1){2}}
\put(2.5,1){\line(0,1){2}}
\put(4,1){\line(0,1){2}}
\put(1.75,2.5){\makebox(0,0){$\re_1$}}
\put(3.25,2.5){\makebox(0,0){$\re_2$}}
\put(1.75,1.5){\makebox(0,0){$\re_3$}}
\put(3.25,1.5){\makebox(0,0){$\re_4$}}
\put(2.5,0.5){\vector(1,0){1.5}}
\put(2.5,0.5){\vector(-1,0){1.5}}
\put(2.5,0.25){\makebox(0,0){$2h_x$}}
\put(1,0.25){\line(0,1){0.5}}
\put(4,0.25){\line(0,1){0.5}}
\put(4.5,2){\vector(0,1){1}}
\put(4.5,2){\vector(0,-1){1}}
\put(5,2){\makebox(0,0){$2h_y$}}
\put(4.25,1){\line(1,0){0.5}}
\put(4.25,3){\line(1,0){0.5}}
\end{picture}
\end{center}
\caption{Søskende sæt\label{siblingset}}
\end{figure} 

Målet med netinddelings algoritmen er at konstruere en tilladt
netinddeling, hvor $\kappa_1(\re)=\kappa_2(\re)=1$ for alle elementer
$\re$, dvs. alle elementer har den ``rigtige'' tæthed. Fra
ligningerne~\eqref{kappa1} og \eqref{kappa2} ses, at en forfining eller
forgrovning i $x$-retningen (dvs. en forøgelse eller formindskning af $h_1$)
forøger eller formindsker $\kappa_1$ uden af der (op til højere ordens led)
ændres på $\kappa_2$. Tilsvarende overvejelser gør sig gældende om
forfining/forgrovning i $y$-retningen. 

For at udlede en heuristik for netinddelings algoritmen skal vi se
bort fra højere ordens led samt antage at net størrelse funktionerne
$\hat{h}_1$ og $\hat{h}_2$ er konstante. Dette svarer til situationen,
hvor net størrelses funktionerne (og dermed net tætheds funktionen) er
glatte og uden singulariteter. 

\subsubsection{Netforfining} 
Som vist i figur~\ref{pos-subdivisions}(a) kan et element $\re$ med
sidelængder $h_1\times h_2$ forfines på tre forskellige måder.  

Er $1<\kappa_1(\re)<2$ skal $h_1$ formindskes, og er $\kappa_1(\re)>2$, skal
$h_1$ halveres. Lignende observation gør sig gældende for $\kappa_2$
og $h_2$. Dette giver os følgende heuristik.
 \begin{ttitemize}
  \item  Hvis $\kappa_1(\re)>\beta$ halveres $h_1$
  \item  Hvis $\kappa_2(\re)>\beta$ halveres $h_2$   
\end{ttitemize}
hvor $1\leq\beta\leq 2$. Sættes $\beta =1$, er det muligt, at vi får
en over-forfining af netinddelingen, idet vi så halverer $h_1$ $(h_2)$
selvom $\kappa_1(\re)$ $(hhv. \kappa_2(\re))$ kun er en anelse større end
1. Tilsvarende kan $\beta =2$ føre til en under--forfining, idet vi her
kun halverer $h_1$ (hhv. $h_2$), når $\kappa_1(\re)$ (hhv. $\kappa_2(\re)$) er
dobbelt så stor, som den burde være.

\subsubsection{Netforgrovning}
Som vi nu skal se, er situationen med netforgrovning en anelse mere
kompliceret. Igen er der tre tilfælde svarende til søskende sæt af
henholdsvis type $x$, $y$ og $xy$. Notationen er som i
figur~\ref{pos-subdivisions}(a).  

Lad os starte med type $x$. Er $\kappa_1(\re_1\cup\re_2)<1$ slås
$\re_1$ og $\re_2$ sammen til et element. Har vi derimod
$\kappa_1(\re_1)>1$ og $\kappa_1(\re_2)>1$ eller ækvivalent
$\kappa_1(\re_1)+\kappa_1(\re_2)>2$ ($\hat{h}_1$ er antaget konstant)
gør vi ingenting. Ovenstående diskussion kan gentages for søskende sæt
af type $y$, hvor $\kappa_1$ er erstattet med $\kappa_2$. Denne
diskussion kan sammenfattes i heuristikken

\begin{ttitemize}
  \item For søskende sæt $(\re_1,\re_2)$ af type $x$ \\
        \makebox[.25in][r]{} Hvis $\kappa_1(\re_1)+\kappa_1(\re_2)<\alpha$
        fordobles $h_1$
  \item For søskende sæt $(\re_1,\re_2)$ af type $y$ \\
        \makebox[.25in][r]{} Hvis $\kappa_2(\re_1)+\kappa_2(\re_2)<\alpha$   
        fordobles $h_2$
\end{ttitemize}
Her er $1\leq\alpha\leq 2$. Vælges $\alpha =1$, er det muligt med
over-forfining, idet vi kun forgrover netinddelingen, når det er absolut
nødvendigt. For $\alpha =2$ kan vi risikere en under-forfining, idet
vi i denne situation forgrover, selvom elementerne kun er en anelse for
små.

Vi mangler nu kun søskende sæt af type $xy$. Dette er den mest
komplicerede situa\-tion på grund af de mange muligheder for forgrovning. I
figur~\ref{lov-forgrovninger} ses alle tilladte forgrovninger samt et
eksempel på en ikke-tilladt netinddeling. Ved en tilladt
forgrovning forstås en forgrovning, der resulterer i en tilladt netinddeling. 
\begin{figure}[htb]
\begin{center}
\setlength{\unitlength}{1cm}
\begin{picture}(14.5,5.5)(0,0)
\put(0.5,3){
\begin{picture}(3.5,2)(0,0)
\put(0,0){\line(1,0){2}}
\put(0,0.75){\line(1,0){2}}
\put(0,1.5){\line(1,0){2}}
\put(0,0){\line(0,1){1.5}}
\put(1,0){\line(0,1){1.5}}
\put(2,0){\line(0,1){1.5}}
\put(0.5,1.125){\makebox(0,0){$\re_1$}}
\put(1.5,1.125){\makebox(0,0){$\re_2$}}
\put(0.5,0.375){\makebox(0,0){$\re_3$}}
\put(1.5,0.375){\makebox(0,0){$\re_4$}}
\put(2.25,0.75){\vector(1,0){1}}
\end{picture}
}
\put(4,3){
\begin{picture}(2.5,2)(0,0)
\put(0,0){\line(1,0){2}}
\put(0,1.5){\line(1,0){2}}
\put(0,0){\line(0,1){1.5}}
\put(2,0){\line(0,1){1.5}}
\put(2.25,0.75){\makebox(0,0){,}}
\put(1,1.75){\makebox(0,0){tilfælde 1}}
\end{picture}
}
\put(6.5,3){
\begin{picture}(2.5,2)(0,0)
\put(0,0){\line(1,0){2}}
\put(0,1.5){\line(1,0){2}}
\put(0,0){\line(0,1){1.5}}
\put(1,0){\line(0,1){1.5}}
\put(2,0){\line(0,1){1.5}}
\put(2.25,0.75){\makebox(0,0){,}}
\put(1,1.75){\makebox(0,0){tilfælde 2}}
\end{picture}
}
\put(9,3){
\begin{picture}(2.5,2)(0,0)
\put(0,0){\line(1,0){2}}
\put(0,0.75){\line(1,0){2}}
\put(0,1.5){\line(1,0){2}}
\put(0,0){\line(0,1){1.5}}
\put(2,0){\line(0,1){1.5}}
\put(2.25,0.75){\makebox(0,0){,}}
\put(1,1.75){\makebox(0,0){tilfælde 3}}
\end{picture}
}
\put(11.5,3){
\begin{picture}(2.5,2)(0,0)
\put(0,0){\line(1,0){2}}
\put(1,0.75){\line(1,0){1}}
\put(0,1.5){\line(1,0){2}}
\put(0,0){\line(0,1){1.5}}
\put(1,0){\line(0,1){1.5}}
\put(2,0){\line(0,1){1.5}}
\put(1,1.75){\makebox(0,0){tilfælde 4}}
\end{picture}
}
\put(4,0.5){
\begin{picture}(2.5,2)(0,0)
\put(0,0){\line(1,0){2}}
\put(0,0.75){\line(1,0){1}}
\put(0,1.5){\line(1,0){2}}
\put(0,0){\line(0,1){1.5}}
\put(1,0){\line(0,1){1.5}}
\put(2,0){\line(0,1){1.5}}
\put(2.25,0.75){\makebox(0,0){,}}
\put(1,1.75){\makebox(0,0){tilfælde 5}}
\end{picture}
}
\put(6.5,0.5){
\begin{picture}(2.5,2)(0,0)
\put(0,0){\line(1,0){2}}
\put(0,0.75){\line(1,0){2}}
\put(0,1.5){\line(1,0){2}}
\put(0,0){\line(0,1){1.5}}
\put(1,0.75){\line(0,1){0.75}}
\put(2,0){\line(0,1){1.5}}
\put(2.25,0.75){\makebox(0,0){,}}
\put(1,1.75){\makebox(0,0){tilfælde 6}}
\end{picture}
}
\put(9,0.5){
\begin{picture}(2.5,2)(0,0)
\put(0,0){\line(1,0){2}}
\put(0,0.75){\line(1,0){2}}
\put(0,1.5){\line(1,0){2}}
\put(0,0){\line(0,1){1.5}}
\put(1,0){\line(0,1){0.75}}
\put(2,0){\line(0,1){1.5}}
\put(1,1.75){\makebox(0,0){tilfælde 7}}
\end{picture}
}
\put(11.5,0.5){
\begin{picture}(2.5,2)(0,0)
\put(0,0){\line(1,0){2}}
\put(1,0.75){\line(1,0){1}}
\put(0,1.5){\line(1,0){2}}
\put(0,0){\line(0,1){1.5}}
\put(1,0){\line(0,1){0.75}}
\put(2,0){\line(0,1){1.5}}
\put(1,1.75){\makebox(0,0){ikke-tilladt}}
\end{picture}
}
\end{picture}
\end{center}
\caption{Tilladte forgrovninger samt et eksempel på en ikke-tilladt
forgrovning\label{lov-forgrovninger}}
\end{figure}

I tilfælde 1 slås alle 4 elementer i søskende sættet sammen til et
element. Dette gøres når
$\kappa_i(\re_1\cup\re_2\cup\re_3\cup\re_4)<1$ for $i=1,2$. Er
$\kappa_i(\re_1)$, $\kappa_i(\re_2)$, $\kappa_i(\re_3)$, $\kappa_i(\re_4)>1$
for $i=1,2$ foretages intet. Anvendes~\eqref{kapparelations}, samt det faktum at
$\kappa_i(\re_j)$ er uafhængig af $j$ for konstant $\hat{h}_i$, fås
første del af netforgrovnings heuristikken for søskende sæt af type
$xy$ til 
\begin{ttitemize}
  \item For søskende sæt $(\re_1,\re_2,\re_3,\re_4)$ af type $xy$ \\
        \makebox[.25in][r]{} Hvis $\sum_{j=1}^4 \kappa_i(\re_j)<2\alpha$
        for $i=1,2$ slå \\
        \makebox[.25in][r]{} da alle 4 elementer sammen til 1 element   
\end{ttitemize}
Som ovenfor er $1\leq\alpha\leq 2$.

I tilfældene 2--7 skal vi være mere påpasselige for at undgå
ikke-tilladte forgrovninger. Ideen er her forgrove de elementer, hvor
det er mest nødvendigt, og derefter forsætte med at forgrove så længe
dette ikke giver ikke-tilladte netinddelinger.  

Fra figur~\ref{siblingset} ses, at eventuelle forgrovning i disse
tilfælde kontrolleres af tallene
$\kappa_1(\re_1)+\kappa_1(\re_2)$, $\kappa_1(\re_3)+\kappa_1(\re_4)$,
$\kappa_2(\re_1)+\kappa_2(\re_3)$ og $\kappa_2(\re_2)+\kappa_2(\re_4)$.  
Vi definerer nu $\kappa^{(1)}$ som værende det største af disse fire tal,
og $\kappa^{(2)}$ som værende det tal med samme indeks på $\kappa$ som
$\kappa^{(1)}$. Er $\kappa^{(1)}<\alpha$ slås de to involverede
elementer sammen. Dette vil altid give et tilladt net. Er derimod
$\kappa^{(1)}>\alpha$ lader vi de to involverede elementer være
uændrede, og forgrovnings processen stoppes. Ellers fortsættes med
$\kappa^{(2)}$. Har vi $\kappa^{(2)}<\alpha$
forgrover vi og ellers foretages intet. Dette giver ligeledes en
tilladt netinddeling. Er $\kappa^{(1)}$ valgt som  
$\kappa_2(\re_1)+\kappa_2(\re_3)$ eller $\kappa_2(\re_2)+\kappa_2(\re_4)$
fås et af tilfældene 2, 4 eller 5. Er $\kappa^{(1)}$ derimod valgt som 
$\kappa_1(\re_1)+\kappa_1(\re_2)$ eller $\kappa_1(\re_3)+\kappa_1(\re_4)$
haves et af tilfældene 3, 6 eller 7. Dette giver den resterende del af
heuristikken for forgrovning af søskende sæt af type $xy$. Enhver
yderligere forgrovning vil føre til en ikke-tilladt netinddeling
(eller til tilfælde 1, hvilket allerede er dækket).  
\begin{ttitemize}
  \item For søskende sæt $(\re_1,\re_2,\re_3,\re_4)$ af type $xy$ \\
        \makebox[.25in][r]{} Lad $\kappa^{(1)}$ og $\kappa^{(2)}$ være
          som ovenfor, og antag \\
        \makebox[.25in][r]{} at $\kappa^{(1)}=\kappa_{i_1}(\re_{j_1}) + 
          \kappa_{i_1}(\re_{j_2})$ og $\kappa^{(2)}=\kappa_{i_1}(\re_{j_3}) + 
          \kappa_{i_1}(\re_{j_4})$ \\
        \makebox[.25in][r]{} (Her er $i_1\in\{1,2\},\ \{j_1,j_2,j_3,j_4\}=
          \{1,2,3,4\}$). \\
        \makebox[.25in][r]{} Hvis $\kappa^{(1)}<\alpha$ slå \\
        \makebox[.5in][r]{} da $\re_{j_1}$ og $\re_{j_2}$ sammen til et element \\ 
        \makebox[.5in][r]{} Hvis $\kappa^{(2)}<\alpha$ slå \\
        \makebox[.75in][r]{} da $\re_{j_3}$ og $\re_{j_4}$ sammen til et element
\end{ttitemize}
Igen er $1\leq\alpha\leq 2$. Som ovenfor kan man også her foretage
betragtninger om over- og under-forfining afhængig af valget af $\alpha$.

\subsubsection{Selve algoritmen}
Ovenfor har vi gennemgået alle tilladte netforfininger og
-forgrovninger. Lad os sammenfatte resultaterne i følgende
netinddelings algoritme. Som input har algoritmen et tilladt net $\T$,
to net størrelses funktioner $\hat{h}_1$ og $\hat{h}_2$ samt
$\hat{N}$, som er det optimale antal af elementer, algoritmen skal
forsøge at inddele domænet i. Desuden skal værdien af de reelle
parametre $\alpha$, $\beta\in[1,2]$ specificeres.
\begin{Algoritme}[H]
\begin{emptylist}
\setlength{\itemsep}{0cm}
\setlength{\parsep}{0cm}
  \item Gentag følgende indtil det ikke er muligt \\ 
        \makebox[.25in][r]{} at foretage flere forgrovninger:
  \item For alle søskende sæt gøres følgende:   
  \item For søskende sæt $(\re_1,\re_2)$ af type $x$ \\
        \makebox[.25in][r]{} Hvis $\kappa_1(\re_1)+\kappa_1(\re_2)<\alpha$
        slås $\re_1$ og $\re_2$ sammen til et element
  \item For søskende sæt $(\re_1,\re_2)$ af type $y$ \\
        \makebox[.25in][r]{} Hvis $\kappa_2(\re_1)+\kappa_2(\re_2)<\alpha$   
        slås $\re_1$ og $\re_2$ sammen til et element
  \item For søskende sæt $(\re_1,\re_2,\re_3,\re_4)$ af type $xy$ \\
        \makebox[.25in][r]{} Hvis $\sum_{j=1}^4 \kappa_i(\re_j)<2\alpha$
        for $i=1,2$ \\
        \makebox[.5in][r]{} slå da alle 4 elementer sammen til 1 element
  \item \makebox[.25in][r]{} Ellers \\
        \makebox[.5in][r]{} Lad $\kappa^{(1)}$ og $\kappa^{(2)}$ være
          som ovenfor, og antag \\
        \makebox[.5in][r]{} at $\kappa^{(1)}=\kappa_{i_1}(\re_{j_1}) + 
          \kappa_{i_1}(\re_{j_2})$ og $\kappa^{(2)}=\kappa_{i_1}(\re_{j_3}) + 
          \kappa_{i_1}(\re_{j_4})$ \\
        \makebox[.5in][r]{} (Her er $i_1\in\{1,2\},\ \{j_1,j_2,j_3,j_4\}=
          \{1,2,3,4\}$). \\
        \makebox[.5in][r]{} Hvis $\kappa^{(1)}<\alpha$ \\
        \makebox[.75in][r]{} slå da $\re_{j_1}$ og $\re_{j_2}$ sammen til et element \\ 
        \makebox[.75in][r]{} Hvis $\kappa^{(2)}<\alpha$ \\
        \makebox[1in][r]{} slå da $\re_{j_3}$ og $\re_{j_4}$ sammen til et element
\end{emptylist}
\caption{Forgrovnings delen af algoritmen\label{forgrov}}
\end{Algoritme}
\begin{Algoritme}[H]
\begin{emptylist}
\setlength{\itemsep}{0cm}
\setlength{\parsep}{0cm}
  \item Gentag følgende indtil det ikke er muligt \\ 
        \makebox[.25in][r]{} at foretage flere forfininger:
  \item For alle elementer $\re$ i $\T$
  \item \makebox[.25in][r]{} Hvis $\kappa_1(\re)>\beta$ halveres $h_1$
  \item \makebox[.25in][r]{} Hvis $\kappa_2(\re)>\beta$ halveres $h_2$
\end{emptylist}
\caption{Forfinings delen af algoritmen\label{forfin}}
\end{Algoritme}

Benyttes kvadratiske elementer, er netinddelingen (i lighed med
\eqref{kappa1} og \eqref{kappa2}) kontrolleret af funktionen 
\begin{equation}
  \kappa (\re ) = \frac{h}{\hat{h}(\overline{x},\overline{y})} =
  \frac{1}{h} \int_\re \frac{1}{\hat{h}}\, dx\, dy + \hot
\end{equation}
I analogi med tidligere er målet her at opnå $\kappa(\re )=1$ for alle
$\re$ i netinddelingen. Da netforfining og -forgrovning skal give
lovlige netinddelinger, er vi i det kvadratiske tilfælde nødsaget til
kun at bruge forfininger og forgrovninger af type $xy$, men ellers kan
algoritmen ovenfor anvendes uændret. 

\subsection{Teoretiske resultater for algortimen} \label{teorires}
I artiklen~\cite{hugger-net} findes omfattende numeriske test af
algoritmen. Fra et matematisk synspunkt er det dog ikke
tilfredsstillende blot at have gode praktiske erfaringer med en given
algoritme. Vi skal derfor i dette afsnit redegøre for nogle teoretiske
resultater vedrørende algoritmens virkemåde. 

Lad $\hat{N}$ være det teoretisk optimale antal elementer i
netinddelingen, der skal til for at en given forudbestemt
brugertolerance nås (se afsnit~\ref{noae}), og lad $N$ være det faktiske antal elementer
opnået ved at anvende algoritmen. Hvis $N>\hat{N}$ siger vi at
algoritmen har resulteret i en over-forfining. Har vi omvendt
$N<\hat{N}$ taler vi om en under-forfining. Den optimale situation er
selvfølgelig at have $N=\hat{N}$, men dette ses sjældent i praksis.
Det er derfor vigtigt at have nedre og øvre grænser for $N/\hat{N}$,
hvilket nedenstående sætning blandt andet giver  

\begin{theorem}{(Over- og underforfining)} \label{theoryalg}
Antag at a priori fejlligningen~\eqref{fejlligning} holder, og at summations
princippet~\eqref{sumprincip} gælder. Med $\hat{h}_1$ og $\hat{h}_2$ betegnes de
asymptotisk optimale net størrelses funktioner givet ved
ligningerne~\eqref{opt-h1} og \eqref{opt-h2}. Funktionerne $\hat{h}_1$
og $\hat{h}_2$ antages at have højest algebraisk singulariteter, at
være kontinuerte samt at have begrænsede afledede. Lad $\hat{N}$ være
det minimale antal af elementer, der skal til for at opfylde kravet
til den på forhånd givne fejltolerance. Den optimale net tætheds
funktion betegnes så med $\hat{D}=1/(\hat{N}\hat{h}_1\hat{h}_2)$. For
ethvert rektangulært element $\re$ med sidelængder $h_1$ og $h_2$ i
hhv. $x$ og $y$ retningen i en tilladt netinddeling defineres
$\kappa_1(\re)=h_1/\hat{h}_1(\overline{x},\overline{y})$ og
$\kappa_2(\re)=h_2/\hat{h}_2(\overline{x},\overline{y})$ som i
ligningerne~\eqref{kappa1} og \eqref{kappa2}. Punktet
$(\overline{x},\overline{y})$ er her et vilkårligt punkt i $\re$.
Endelig er $\alpha$, $\beta\in [1,2]$ de parametre, der anvendes i
algoritmerne~\eqref{forgrov} hhv. \eqref{forfin}, og $N$ er det antal
elementer algoritmene resulterer i (med ovennævnte input). Da gælder
følgende relationer 
\begin{equation} \label{onlychange}
  \frac{\beta}{2} \leq \kappa_1(\re), \ \kappa_2(\re) \leq \beta ,
  \qquad (\text{for $h_1$, $h_2\rightarrow 0$}),
\end{equation}
og
\begin{equation}
  \frac{\beta^2}{4} \leq \hat{N} \int_\re \hat{D}(x,y)\, dx\, dy
  \leq \beta^2 , \qquad (\text{for $h_1$, $h_2\rightarrow 0$}),
\end{equation}
\begin{equation}
  \frac{1}{4} \leq N \int_\re \hat{D}(x,y)\, dx\, dy \leq 4 ,  
  \qquad (\text{for $h_1$, $h_2\rightarrow 0$}),
\end{equation}
\begin{equation}
  \frac{1}{\beta^2} \leq \frac{N}{\hat{N}} \leq \frac{4}{\beta^2}
  \qquad (\text{for $h_1$, $h_2\rightarrow 0$}).
\end{equation}
\end{theorem}
\begin{remark}
Arbejdes der med kvadratiske elementer $\re$ med sidelængde $h$, har
vi naturligvis kun en asymptotisk optimal netstørrelses funktion
$\hat{h}$, en $\kappa$-funktion
$\kappa(\re)=h/\hat{h}(\overline{x},\overline{y})$, og den
asymptotiske net tætheds funktion er så givet ved
$\hat{D}=1/(\hat{N}\hat{h}^2)$. Da gælder tilsvarende resultater til
ovenstående sætning.
Eneste ændring er i ulighed~\eqref{onlychange}, som skal erstattes af 
\begin{equation}
  \frac{\beta}{2} \leq \kappa(\re) \leq \beta ,
  \qquad (\text{for $h\rightarrow 0$}).
\end{equation} 
\end{remark}
\begin{proof}
Der henvises til artiklen~\cite{hugger-2}, hvor sætningen er vist for
kvadratiske elementer.
\end{proof}

Sætning~\ref{theoryalg} viser, at forholdet $N/\hat{N}$ asymptotisk er
under kontrol. 
\begin{remark} \label{smallremark}
Ligningen~\eqref{onlychange} svarer til ligningen~\eqref{approxkK} med
$k=\beta /2$ og $K=\beta$. 
\end{remark}
Sammenhængen mellem algoritmen og den globale fejl er indeholdt i
følgende sætning.
\begin{theorem}{(Den globale fejl)}
Antag at forudsætningerne fra sætning~\ref{theoryalg} gæl\-der, samt at
notationen er uændret. Da haves følgende globale grænser for fejlen
\begin{equation}
  (\beta/2)^{\gamma} \cdot \tau^q \leq \|e\|_{\Omega}^q 
  \leq \beta^\gamma \cdot \tau^q, 
  \qquad (\text{for $h_1$, $h_2\rightarrow 0$}).
\end{equation}
I denne ligning er $\tau$ den brugerbestemte fejltolerance, der også
optræder i formel~\eqref{valueofN} for $\hat{N}$.
\end{theorem}
\begin{proof}
Fra ligningerne~\eqref{fejlligning} og \eqref{approxkK} haves
\begin{align}
  \|e\|_{\re_j}^q &= \phi_1^q h_{1,j}^{\gamma +1} h_{2,j} +
    \phi_2^q h_{1,j} h_{2,j}^{\gamma +1} \\
  &= \phi_1^q \frac{h_{1,j}^{\gamma}}{h_1^\gamma} 
    h_{1,j}h_{2,j}h_1^\gamma +
    \phi_2^q \frac{h_{2,j}^{\gamma}}{h_2^\gamma} 
    h_{1,j}h_{2,j}h_2^\gamma \notag \\
  &\leq \phi_1^q h_1^\gamma h_{1,j}h_{2,j} K^\gamma +
    \phi_2^q h_2^\gamma h_{1,j}h_{2,j} K^\gamma \notag \\
  &= \int_{\re_j} (\phi_1^q h_1^\gamma + \phi_2^q h_2^\gamma)
    \, dx\, dy\, K^\gamma . \notag     
\end{align}
Ved nu at anvende betingelse~\eqref{cononN},
be\-mærk\-ning~\ref{smallremark} samt
sum\-ma\-tions\-prin\-cip\-pet~\eqref{sumprincip} fås sætningens ene ulighed. Den
anden følger på tilsvarende vis. 
\end{proof}

Følgende lokale version af ovenstående sætning viser, at algoritmen
genererer asymptotisk ækvilibrerede netinddelinger ud fra asymptotisk
optimale net størrelses funktioner. Ved en ækvilibreret netinddeling
forstås en netinddeling, hvor den lokale fejl er ens i alle
elementerne i netinddelingen. 
\begin{theorem}{(Den lokale fejl)}
Antag at forudsætningerne fra sætning~\ref{theoryalg} gæl\-der, samt at
notationen er uændret. Da haves følgende lokale grænser for fejlen
\begin{equation}
  (\beta/2)^{\gamma} \cdot 2C \leq \|e\|_{\re}^q 
  \leq \beta^\gamma \cdot 2C, 
  \qquad (\text{for $h\rightarrow 0$}),
\end{equation}
hvor værdien af konstanten $C$ er givet ved formel~\eqref{valueofC}.
\end{theorem}
\begin{proof}
Den eneste forskel fra beviset for forrige sætning er, at man her
anvender ligning~\eqref{error-eq-dis} istedet for \eqref{approxkK}.
\end{proof}






