\pagenumbering{arabic}
\chapter{Introduktion}

Der kan gives mange eksempler fra fx fysik og økonomi på problemer,
der resulterer i et ønske om at løse differentialligninger af typen
\begin{equation} \label{sec:intro-eq}
{\mathcal D}u = f,
\end{equation}
hvor $\mathcal D$ er en given differentialoperator, muligvis med rand-
eller be\-gyn\-del\-ses\-vær\-di\-be\-tin\-gel\-ser, $f$ en given
funktion og $u$ den
ukendte løsning til problemet. Ofte kan man ikke klare sig med
analytiske metoder til løsning af ligningen \eqref{sec:intro-eq}, og
man vil derfor anvende numeriske metoder. En af disse metoder kaldes
Galerkins metode. I dette kapitel skal vi give en overordnet
beskrivelse af denne, men først et par historiske og bibliografiske noter.

\section{Historiske og bibliografiske noter}
I forbindelse med numerisk løs\-ning af partielle differentialligninger
begyndte in\-gen\-iører fra flyindustrien i USA i slutningen af 50'erne at
anvende en metode, der ligner den metode, vi i dag kender som den
endelige element metode. Da man i midten af 60'erne begyndte at studere den
matematiske teori bag metoden, blev det klart, at der var tale om en
generel metode til numerisk løsning af differential- og
integralligninger. Op gennem 70'erne blev den matematiske teori for
især $h$-metoden grundlagt, og i 80'erne kom så $p$-metoden til. I
$h$-metoden  forfines ind\-de\-ling\-en af domænet yderligere,
såfremt en given fejltolerance ikke nås. I $p$-metoden
fastholder man derimod netinddelingen og øger i stedet graden af de
polynomier, som man forsøger at approksimere løsningen med.
Desuden var og er fejlestimation en stadig kilde til
forsk\-ning. I de senere år har man yderligere arbejdet med $hp$-metoden, dvs en
kombination af $h$- og $p$-metoden. Ved udgangen af 80'erne var
hovedparten af teorien for endimensionelle problemer på plads, mens
der stadigvæk er åbne spørgsmål for to- og tredimensionelle problemer.

En standard lærebog om den endelige element metode er
\cite{ciarlet78}. Kapitlet om variations former er inspireret af denne
bog samt af \cite{babuska82}. Endelige elementer og endelige element
rum er ligeledes godt beskrevet i \cite{ciarlet78}.
Funktionsinterpolation er gennemgået meget generelt i
\cite{ciarlet78}. Vi har brug for et mere specielt tilfælde, som er
bedre beskrevet i \cite{hugger-pol}. Netinddelings algoritmen vi skal studere
senere er kun en enkelt algoritme blandt mange. Artiklen \cite{hugger-net}
indeholder referencer til andre algoritmer. Fejlestimering er et
område med en del forskningsaktivitet. I
\cite{babuska1,babuska2,babuska3} findes omfattende
litteraturhenvisninger til artikler om dette emne.

\section{Galerkins metode}
For at løse ligningen \eqref{sec:intro-eq} vælger vi en række lineært
uafhængige løsnings funktioner (eng. trial functions)
$\phi_1,\ldots,\phi_m$ og forsøger at tilnærme $u$ med en
linearkombination af disse, dvs.
\begin{equation} \label{sec:intro-approx}
u = \sum_{j=1}^m \xi_j \phi_j.
\end{equation}
Indsættes denne approksimation i differentialligningen
\eqref{sec:intro-eq} fås et residual givet ved
\begin{equation}
{\mathcal D}\bigl(\sum_{j=1}^m \xi_j \phi_j\bigr) - f.
\end{equation}
For at få så god en approksimation til $u$ som muligt, ønsker vi, at
dette residual skal være så lille som muligt. Vi vil derfor kræve, at
integralet af residualet vægtet med en række af test funktioner (eng.
test functions) er $0$. Ideen bag dette krav er hentet fra indre
produkter på funktionsvektorrum. I ${\mathcal L}^2$ har vi det indre produkt
$\inner{f}{g}=\int_\dom f(x)g(x)\, dx$ for to funktioner $f$ og $g$. 
Som bekendt siger vi, at $f$ og $g$ er ortogonale, hvis $\inner{f}{g}=0$. 
Den geometriske fortolkning er her, at vi, som approksimeret
løsning, vælger den løsning, der ligger ``tættest'' på testrummet
(dvs. den løsning som er ortogonal på testrummet).

Approksimationen \eqref{sec:intro-approx} indeholder $m$ ubekendte
konstanter $\xi_1,\ldots,\xi_m$. Ved at vælge $m$ lineært uafhængige
test funktioner $w_1,\ldots,w_m$ fås et $m\!\times\!m$ lineært
ligningssystem
\begin{equation}
\sum_{j=1}^m \xi_j \int {\mathcal D} \phi_j w_i\, dx = 
\int fw_i\, dx, \qquad 1\leq i\leq m,
\end{equation}
hvorfra $\xi_1,\ldots,\xi_m$ kan bestemmes. Denne metode er kendt som
den vægtede re\-si\-du\-al metode. Vælges $\phi_1,\ldots,\phi_m$ som test
funktioner kaldes den ovenfor beskrevne fremgangsmåde for Galerkins
metode. 
  
\section{Et eksempel}
Som et konkret eksempel betragter vi følgende model problem: Bestem
$u$ således at
\begin{align} \label{sec:eks-problem}
  -u'' + u &  = f, \qquad \text{for $x\in]0,1[$,}  \\ 
      u(0) &  = 0, \qquad u(1) = 0, \notag
\end{align}
hvor $f$ er en given funktion. Vælges test og løsnings funktionerne som
kontinuerte stykkevise polynomier kaldes Galerkins metode også for den
endelig element metode. Vi skal i dette eksempel benytte kontinuerte
stykkevis lineære funktioner som test og løsnings funktioner. 

Lad $\T_h: 0=x_0 < x_1 < \dots < x_{m+1}=1$ være en inddeling af
$I=]0,1[$ i elementer $I_j=]x_{j-1},x_j[$ med længde
$h_j=x_j-x_{j-1}$, og lad ${\mathcal X}_{0,{\mathsf{fe}}}$ være 
vektorrummet bestående af kontinuerte stykkevis lineære funktioner 
på $\T_h$, som er 0 i punkterne $x=0$ og $x=1$. Intervallerne
$I_j=]x_{j-1},x_j[$, $j=1,\ldots,m+1$ kaldes for endelige elementer.
(Vi skal senere give en præcis definition.) Defineres 
$\phi_j, \ j=1,\ldots,m$ som kontinuerte stykkevis lineære funktioner
på $\T_h$ opfyldende
\begin{equation}
  \phi_j(x_i)=
    \begin{cases}
      1 & \text{hvis $i=j$}, \\
      0 & \text{ellers},
    \end{cases}
\end{equation}
er det oplagt, at $\phi_j \in {\mathcal X}_{0,{\mathsf{fe}}}$, og at 
$\{\phi_j\}_{j=1}^m$ er en basis for ${\mathcal X}_{0,{\mathsf{fe}}}$,
thi enhver funktion i ${\mathcal X}_{0,{\mathsf{fe}}}$ er entydig
bestemt ved dens værdi i de indre punkter $x_j, \ j=1,\ldots,m$.

For at anvende Galerkins metode giver vi \eqref{sec:eks-problem}
en variationel formulering: Find en funktion $u\in\H^1_0([0,1])$,
således at
\begin{equation} \label{sec:eks-variationel}
  \int_I \bigl( u'v' +uv \bigr)\, dx = \int_I fv\, dx 
\end{equation} 
for alle funktioner $v\in\H^1_0([0,1])$. Denne ligning er fremkommet
ved at multiplicere \eqref{sec:eks-problem} med funktionen $v$ og
dernæst foretage en partiel integration over $I$ for integralet
\begin{equation}
  - \int_I u''v\, dx = \bigl[-u'v\bigr]_0^1 + \int_I u'v'\, dx = 
    \int_I u'v'\, dx
\end{equation}
idet $v(0)=v(1)=0$.

En (analog) endelig dimensional formulering af
\eqref{sec:eks-variationel} giver os den endelige element metode 
for \eqref{sec:eks-problem}: Find $\ufe \in {\mathcal X}_{0,{\mathsf{fe}}}$                 
således at 
\begin{equation} \label{sec:eks-fem}
  \int_I \bigl( \ufe'v' +\ufe v\bigr)\, dx = \int_I fv\, dx , 
  \quad \text{for alle} \ v \in {\mathcal X}_{0,{\mathsf{fe}}}.
\end{equation}
Ved nu at bemærke, at 
${\mathcal X}_{0,{\mathsf{fe}}}\subset\H^1_0([0,1])$ ses,
at~\eqref{sec:eks-variationel} gælder for $v\in{\mathcal
X}_{0,{\mathsf{fe}}}$. Ved at subtrahere~\eqref{sec:eks-fem} fra
\eqref{sec:eks-variationel} fås følgende relation for $v\in{\mathcal
X}_{0,{\mathsf{fe}}}$ 
\begin{equation} \label{defg}
  \int_I \bigl( (u'-\ufe')v' + (u-\ufe)v\bigr)\, dx = 0.
\end{equation}    
Ligningen~\eqref{defg} genkender vi som det indre produkt på $\H^1_0$
af $u-\ufe$ og $v$.
Med andre ord fejlen $u-\ufe$ er ortogonal på ${\mathcal
X}_{0,{\mathsf{fe}}}$ med hensyn til det indre produkt på $\H^1_0$.
Dette kan vi også udtrykke som følger: Den endelige element løsning
$\ufe$ er projektionen (med hensyn til det indre produkt på $\H^1_0$)
af den eksakte løsning $u$ på ${\mathcal X}_{0,{\mathsf{fe}}}$, dvs.
$\ufe$ er den funktion i ${\mathcal X}_{0,{\mathsf{fe}}}$, som ligger
tættest på $u$ målt i $\H^1_0$ normen. Symbolsk kan dette illustreres
som vist i figur~\ref{geometrisk}, hvor $\H^1_0$ er repræsenteret som
hele planen, og den rette linie gennen centrum repræsenterer
${\mathcal X}_{0,{\mathsf{fe}}}$. 
\setlength{\unitlength}{1mm}
\begin{figure}[htb]
\begin{center}
\begin{picture}(50,60)(0,0)
\put(5,2){\vector(0,1){48}}
\put(2,5){\vector(1,0){58}}
\put(55,30){\line(-4,-2){54}}
\dashline{2}(25,35)(33,19)
\put(15,45){\makebox(0,0){$\H^1_0$}}
\put(57,32){\makebox(0,0){${\mathcal X}_{0,{\mathsf{fe}}}$}} 
\put(25,37){\makebox(0,0){$u$}}
\put(35,17){\makebox(0,0){$\ufe$}}
\thicklines
\put(5,5){\vector(2,3){20}}
\put(5,5){\vector(4,2){28}}
\end{picture}
\end{center}
\caption{Geometrisk fortolkning af den endelige element løsning\label{geometrisk}}
\end{figure}

Da $\ufe \in {\mathcal X}_{0,{\mathsf{fe}}}$, kan $\ufe$ skrives på formen
\begin{equation}
  \ufe = \sum_{j=1}^m \xi_j \phi_j .
\end{equation}
Koefficienterne $\xi_j$, $j=1,\ldots ,m$ kan bestemmes ved i 
\eqref{sec:eks-fem} på skift at vælge $v=\phi_i$, $i=1,\ldots,m$. 
Herved fås et $m\!\times\!m$ lineært ligningssystem
\begin{equation} \label{des1}
  \mathbf{A}\xi = b,
\end{equation}
hvor $\mathbf{A}$ er en $m\!\times\!m$ matrix med elementer
\begin{equation} \label{des2}
  a_{ij} = \int_I \bigl( \phi_j^{'}\phi_i^{'} + \phi_j\phi_i\bigr)\, dx,
\end{equation}
og $b$ en vektor med elementer
\begin{equation} \label{des3}
  b_j = \int_I f\phi_j\, dx,
\end{equation} 
med $\xi$ som løsning. Matricen $\mathbf{A}$ kaldes en stivheds (eng.
stiffness) matrix, og $b$ kaldes en kraft (eng. load) vektor. Vi
bemærker, at $\mathbf{A}$ i dette tilfælde er symmetrisk, positiv
definit og tyndt besat (eng. sparse).

Den grund\-læg\-gen\-de ide i den endelige element metode er alt\-så at sø\-ge
en ap\-prok\-si\-me\-ret løsning i et løsningsrum bestående af polynomier
defineret på en given netinddeling. Den approksimerede løsning vælges nu
således, at den er ortogonal på basisfunktioner for et givet testrum
af polynomier defineret på samme netinddeling. På denne måde fås et 
lineært ligningssystem, hvor løsningen er koefficienterne i den
linearkombination af løsningsfunktioner, der approksimerer den eksakte
løsning.

\section{Den endelige element metode}
Eksemplet i forrige afsnit illustrerer 3 vigtige aspekter ved den
endelige element metode. For det første valg af løsnings- og
testfunktioner. For det andet valg af inddeling af intervallet og for
det tredje omformuleringen af differentialligningen til et
variationsproblem. En mere skematisk oversigt over den endelige
element metode lyder
\begin{enumerate}
  \item Omformulering af differentialligningen til et
        variationsproblem.
  \item Valg af endelige elementer til brug for inddelingen af domænet.
  \item Valg af løsnings- og testrum også kaldet endelige element rum.
  \item Opskrivning og løsning af det tilhørende lineære ligningssystem.
\end{enumerate}
Selve løsningen, af det til den endelige element metode hørende
lineære lignings\-sy\-stem, skal vi ikke beskæftige os nærmere med.
Læseren formodes at være bekendt med metoder hertil fra kurser i lineær algebra
samt kurser i numerisk analyse. Derimod skal vi i de næste 2 kapitler
studere variationsformer og endelige element rum.

Andre mindst lige så vigtige aspekter i den endelige element metode er
funktionsinterpolation, automatisk netinddeling og fejlestimering.
Den netinddelings algoritme, som vi skal studere i
kapitel~\ref{netinddeling}, benytter en ganske
bestemt a priori fejlligning, som derfor vil blive verificeret i
kapitel~\ref{interpolation} om polynomiumsinterpolation. Fejlestimerings
metoder for den endelige element metode kan groft opdeles i to
kategorier: residuum metoder og udglatnings metoder.
Kapitel~\ref{fejlestimering} indeholder en gennemgang af en
fejlestimator af sidstnævnte type.
