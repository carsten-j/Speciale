\chapter*{Noter}
\addcontentsline{toc}{chapter}{Noter}

Afsnittet om historien bag den endelige element metode er koncist men
ufuldstændigt. Dette skal forstås således, at der ikke gives nogle
referencer til artikler, hvori ideerne er præsenteret første gang.
Årsagen til ovenstående valg skyldes (historiske) vanskeligheder med
at datere oprindelse for den endelige element metode samt et, fra min
side, ønske om at undgå kontroverser. For en mere fyldestgørende
redegørelse for historien bag den endelige element metode henvises til
artiklerne~\cite{babuska}, \cite{oden90} og \cite{zienkiewicz72}.  

Af tidsmæssige årsager var det desværre ikke muligt at implementere en
endelig ele\-ment løser eller dele heraf. I stedet har jeg ved at studere
programmerne FEM2D, FEMLAB-ODE~\cite{femlab-ode}, NFEARS~\cite{nfears}
og PLTMG~\cite{pltmg} fået et indblik i, hvordan endelig element
software fungerer. Jeg har desuden haft adgang til dele af koden for
nogle af ovennævnte programmer og har på den måde haft mulighed for at
studere strukturen i sådanne programmer. 

Så godt som al eksisterende litteratur om endelige element metoder er
engelsk sproget. Dette giver af og til vanskeligheder med at finde
passende danske udtryk. I de situationer hvor det har været specielt
svært, har jeg valgt at angive det engelske udtryk i parentes efter
det danske udtryk. En del
af eksemplerne i dette speciale tager udgangspunkt i fysikkens verden.
Min ikke overvældende fysiske indsigt gør, at jeg ligeledes her har valgt at
angive det engelske fagudtryk i parentes efter det danske udtryk. På den
måde skulle eventuelle unøjagtigheder kunne modvirkes. 