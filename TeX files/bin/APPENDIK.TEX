\section{Eksistens og entydigheds s�tninger}
\begin{theorem}[Lax-Milgrams lemma]
Lad $\X$ v�re et Hilbert rum,
$B(\cdot,\cdot):\X\!\times\!\X\rightarrow\R$ en begr�nset $\X$-elliptisk
bilinear form og lad $F:\X\rightarrow\R$ v�re en begr�nset line�r
afbildning. Da har variations problemet: Bestem et element $u\in\X$ s�
\begin{equation} \label{app:var-prob}
B(u,v)=F(v) \qquad \text{for alle $v\in\X$},
\end{equation} 
\'{e}n og kun \'{e}n l�sning.
\end{theorem}
\begin{proof}
Da $B$ er begr�nset findes en konstant $M$ s�
\begin{equation} \label{app:lax-a-kont}
|B(u,v)|\leq M\|u\|\|v\| \qquad \text{for alle $u,v\in V$.}
\end{equation}
For ethvert $u\in\X$ er funktionalen $L:\X\rightarrow\R$ givet ved
$L(v)=B(u,v)$ kontinuert og line�r. Alts� findes et entydigt bestemt
element $x_u\in\X$, se fx \cite[s. 79]{rudin} s�
\begin{equation}
L(v)=\inner{v}{x_u},
\end{equation}
hvor $\inner{\cdot}{\cdot}$ er det indre produkt i $\X$. Et indre
produkt er specielt en line�r funktional, hvorfor der findes et
entydig bestemt element $f_{x_u}\in \dual{\X}$ s�
$\inner{v}{x_u}=f_{x_u}(v)$,  hvor $\dual{\X}$ er det duale rum til $\X$.
Defineres nu $Au=f_{x_u}\in\dual{\X}$ haves
\begin{equation} \label{app:lax-Au-kont}
B(u,v)=L(v)=\inner{v}{x_u}=f_{x_u}(v)=Au(v).
\end{equation} 
Med $\|\cdot \|^{\ast}$ betegnes normen i $\dual{\X}$. Vi har nu fra
\eqref{app:lax-a-kont} og \eqref{app:lax-Au-kont}
\begin{equation}
\|Au\|^{\ast}=\sup_{v\in V\setminus\{ 0\}} 
\frac{|Au(v)|}{\|v\|}\leq M\|u\|,
\end{equation} 
hvilket viser at afbildningen $A:\X\rightarrow \dual{\X}$ er kontinuert
med $\|A\| \leq M$.

I f�lge Riesz' repr�sentations s�tning findes en afbildning 
$\tau : \dual{\X} \rightarrow \X$ s�
\begin{equation}
F(v) = \inner{\tau F}{v} \qquad 
\text{for alle $F \in \dual{\X}$ og alle $v \in\X$.}
\end{equation} 
Vi har nu for $u \in\X$ og for alle $v \in\X$
\begin{equation}
B(u,v) = F(v) \Leftrightarrow
Au(v) = F(v) \Leftrightarrow
\inner{\tau Au}{v} = \inner{\tau F}{v} \Leftrightarrow
\tau Au - \tau F = 0
\end{equation}
At l�se variations problemet \eqref{app:var-prob} er derfor �kvivalent
ved at l�se ligningen $\tau Au = \tau F$.

For at vise at denne ligning har en og kun en l�sning, skal vi se at
afbildningen 
\begin{equation} \label{app:kontraktion}
v \rightarrow v - \rho (\tau Av- \tau F)
\end{equation}
er en kontraktion for passende valgt $\rho$. Da $\X$ er et Hilbert rum,
er $\X$ specielt et fuldst�ndigt metrisk rum, og som bekendt har en
kontraktion defineret p� et fuldst�ndigt metrisk rum netop et
fixpunkt.

Bem�rkes nu at
\begin{gather}
\inner{\tau Av}{v} = Av(v) = B(v,v) \geq \alpha \|v\|^2 \\
\|\tau Av\| = \|Av\|^{\ast} \leq \|A\|\|v\| \leq M\|v\|
\end{gather}
hvor $\alpha$ skyldes $\X$-ellipticitet, f�lger kontraktionen let thi
\begin{align}
\|v-\rho \tau Av\|^2 & = \|v\|^2 - 2\rho \inner{\tau Av}{v}
                         + {\rho}^2 \|\tau Av\|^2 \\
                     & \leq (1-2\rho \alpha + {\rho}^2 M^2) \|v\|^2.
\end{align}
Afbildningen defineret ved \eqref{app:kontraktion} er derfor en
kontraktion for $\rho \in ]0,2\alpha /M^2[$. Alts� findes et
element $v \in V$ s� $v-\rho(\tau Av - \tau F)=v$, dvs $\tau Av = \tau
F$ som �nsket.  
\end{proof}






