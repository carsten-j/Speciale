\chapter{Fejlestimering} 
Når man omformer uendelig-dimensionelle poroblemer i form af
differentialligninger til et sæt af endelig mange ligninger med
endelig mange ubekendte vil der uundgåeligt opstå nogle fejl. En stor
del af forskningsindsatsen indenfor den endelige element metode består
i at udvikle og dokumentere metoder til evaluering af den fejl, der
opstår i denne proces. I dette kapitel skal vi se nærmere på en hjørne
af fejlestimering.

\section{}

\subsection{Beregningsmæssige- og modelleringsfejl}
Den fejl, som man begår ved at omformulere et differentiallignings
problem til et variations problem, kan opdeles i 
\begin{itemize}
  \item beregningsmæssige fejl.
  \item modelleringsfejl.
\end{itemize}
For at konkretisere disse fejl typer skal vi betragte en matematiske model
\begin{equation} \label{math-model}
  \D u=f,
\end{equation}
hvor $\D$ er en differentialoperator (muligvis med begyndelsesværdi-
eller rand\-be\-tin\-gel\-ser) defineret på et vist domæne $\dom$, $f$ en 
given funktion, og $u$ er den ukendte løsning til problemet. Den 
matematiske model består her af $\D$ og $f$. Udgangspunktet for den
matematiske model er typisk en fysisk model\footnote{Ved en fysisk
model forstås en model, der modellere en situation hentet fra fx
økonomi, kemi eller fysik.}
\begin{equation} \label{phys-model}
  \tilde{\D} \tilde{u} = \tilde{f}.
\end{equation}
Her har $\tilde{\D}$, $\tilde{u}$ og $\tilde{f}$ analoge betydninger.
Den fysiske model antages at have en entydig løsning, og vi skal
desuden antage, at den matematiske model er konstrueret på en sådan
måde, at der ligeledes for denne findes en entydig løsning. Lad nu
$\uex$ være løsningen til \eqref{math-model}, og lad $\ufe$ være en approksimeret 
løsning til \eqref{math-model} beregnet ved hjælp af den endelige 
element metode. Vi vil da definere den beregningsmæssige fejl som
\begin{equation}
  e_c = \uex - \ufe .
\end{equation}
Er $\tilde{u}_{\mathsf{ph}}$ løsningen til \eqref{phys-model}, 
defineres modelleringsfejlen som 
\begin{equation}
  e_m = \tilde{u}_{\mathsf{ph}} - \uex .
\end{equation}
Den samlede fejl defineres da som 
\begin{equation}
  e=\tilde{u}_{\mathsf{ph}}-\ufe=\tilde{u}_{\mathsf{ph}}-\uex+\uex-\ufe
  =e_m + e_c ,
\end{equation}
dvs. som summen af modellerings- og beregningsmæssige fejl.
Vi skal her  kun beskæftige os med beregningsmæssige fejl. 

\subsection{Forskellige typer af fejlestimatorer}
Igennem årene er der blevet foreslået et utal at forskellige 
fejlestimatorer til brug i den endelige element metode, se fx
\cite{babuska1} for 
litteraturhenvisninger. Flertallet af disse fejlestimatorer kan 
klassificeres i en af følgende to kategorier:
\begin{itemize}
  \item Udglatnings metoder (eng. smoothening/flux-projection).
  \item Residuum metoder.
\end{itemize}
Vi skal i afsnit~\ref{z2} betragte en bestemt udglatnings metode.
Traditionelt har residuum metoder haft størst udbredelse, men den
udglatnings metode, som vi senere skal studere, vil måske medvirke til at åbne
vejen for en større udbredelse af fejlestimatorer af denne type.
 
\subsection{Forskellige former for fejlestimering}
Man skelner normalt mellem to former for fejlestimering
\begin{itemize}
  \item A priori fejlestimering.
  \item A posteriori fejlestimering.
\end{itemize}
Ved a priori fejlestimering forsøger man at vurdere fejlen mellem den
eksakte løsning og den approksimerede løsning alene på grundlag af
egenskaberne ved den eksakte (ukendte) løsning til problemet. I a 
posteriori fejlestimering forsøger man derimod at vurdere fejlen vha. 
den beregnede approksimerede løsning. De to former for fejlestimering 
ses ofte anvendt i forskellige teoretiske og praktiske situationer. A
priori fejlestimering bruges typisk til at bestemme 
konvergensegenskaberne for den numeriske metode, mens a posteriori
fejlestimering ofte anvendes i adaptive algoritmer som en del af
netinddelings algoritmen og stopkriteriet.

Specifikke egenskaber ved fejlestimatorer er tæt knyttet til den 
type af differentialligninger, som de er konstrueret i tilknytning til. 
Af denne grund findes kun  meget få generelle teoretiske resultater om
fejlestimatorer. Til gengæld findes en del begreber og definitioner,
som man benytter, når man skal vurdere og/eller sammenligne en eller
flere fejlestimatorer. 

\subsection{} \label{def-beg}
\begin{definition}
Ved en celle forstås en mindre samling af sammenhængende elementer, 
se figur \ref{fig-cell}. Bemærk at en celle godt kan bestå af et  
enkelt element.
\end{definition}
\begin{figure}[htbp]
  \setlength{\unitlength}{1bp}
  \begin{center}
    \begin{picture}(168,137)(0,0)
      \put(1,1){\psfig{figure=cell.ps}}
    \end{picture}
  \end{center}
  \caption{Eksempel på celle\label{fig-cell}}
\end{figure}
\begin{definition}
Lad $\dom$ betegne domænet, hvori vi ønsker at løse en given
differentialligning, og lad $\T=\{\fee_k\}_{k=1}^N$ være en
triangulering af $\dom$. Ved en element fejl indikator for elementet
$\fee_k\in\T$ forstås en udtryk, der afhænger af $\ufe$ men ikke
af $\uex$, og som vurderer fejlen i elementet $\fee_k$.
\end{definition}
\begin{example}
Betragtes model problemet \eqref{math-model} med endelig element løsning
$\ufe$, er 
\begin{equation}
  \eta_{\fee} = \meas (\fee)\norm{R_{\fee}(\ufe)}
\end{equation}
et meget simpelt eksempel på en element fejl indikator.
Her er $R_{\fee}(\ufe)=(\D\ufe -f)|_{\fee}$ residualet restringeret til
$\fee$, og $\meas (\fee)$ er arealet af $\fee$.  
\end{example}
\begin{definition}
Lad $\cell$ være en celle. Vi vil da definere en fejlestimator som
\begin{equation}
  \ee = \Bigl( \sum \begin{Sb} 
                      \fee \in \T \\
                      \fee \cap \cell \not= \emptyset
                    \end{Sb}
               \eta_{\fee}^2\Bigr)^{1/2}.
\end{equation}
Her er $\eta_{\fee}$ en element fejl indikator for elementet $\fee$
tilhørende trianguleringen $\T$ af det betragtede domæne.
\end{definition}

For at kunne vurdere hvor god en fejlestimator er, har vi brug for et
mål, der definerer, hvad vi forstår ved en god fejlestimator. Et sådan
mål er effektivitets indekset.

\begin{definition}
Ved effektivitets indekset for en fejlsestimator $\ee$ forstås
\begin{equation}
  \ei = \frac{\ee}{\en{e}},
\end{equation}
hvor $\en{\cdot}$ er den norm, vi ønsker at vurdere forskellen på 
$e=\uex - \ufe$ i. 
\end{definition}

Jo tættere $\ei$ er på $1$, jo bedre er fejlestimatoren $\ee$. 

Normalt forventer man ikke, at fejlen i den endelige element metode er
specielt lille, såfremt netinddelingen er grov. Omvendt bør en god
fejlestimator approksimere den eksakte fejl godt, såfremt
netinddelingen er fin. På den baggrund skal vi nu definere, hvad vi
forstår ved en asymptotisk eksakt fejlestimator.
\begin{definition}
Lad $\ee$ være en fejlestimator. $\ee$ siges at være asymptotisk
eksakt, såfremt
\begin{equation}
  \ei \rightarrow 1 \quad \text{for $\norm[\cell]{e}\rightarrow 0$}
\end{equation}
for en given klasse af løsninger og en følge af netinddelinger.
\end{definition}