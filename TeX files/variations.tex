\chapter{Variatiationsformer} \label{variation}
Et karakteristikum ved den endelige element metode er, at man forsøger
at omformulere et givet differentiallignings problem til et variations
problem. Vi skal i dette kapitel blandt andet forklare, hvilke fordele der er ved
at gå fra et differentiallignings problem til et variations problem.
Da en sådan omformulering langt fra er entydig, skal vi desuden
forsøge at opstille nogle ``rimelige'' krav til en sådan. De
opstillede krav til variationsformuleringen skal blandt andet sikre,
at der eksisterer en entydig løsning til problemet. I
afsnit~\ref{eks-ent} findes beviser, der godtgør at de opstillede
krav rent faktisk sikrer eksistens og entydighed. I~\cite{babuska82}
kan man finde en række af eksempler, hvor differentiallignings
problemer bliver omformuleret til variations problemer.

\section{Problemstilling}
Betragt problemet: Find $\ucl \in \Cto \cap \C$ så
\begin{quote}
(D) \hspace{.2cm} ${\mathcal{D}} \ucl=f$,
\end{quote}
hvor ${\mathcal{D}}$ er en anden ordens, lineær differentialoperator med rand- eller
be\-gyn\-del\-ses\-vær\-di\-be\-tin\-gel\-ser og f en given funktion. Dette problem vil
vi så gerne omformulere til: Find $\uex \in \Xdom=\X$ så
\begin{quote}
(V) \hspace{.2cm} $B(\uex,v)=F(v)$ for alle $v \in \Ydom=\Y$.
\end{quote}
Her er $\X$ og $\Y$ normerede lineære vektorrum, $B$ en passende bilinear form
defineret på $\X \times \Y$, og $F$ en passende lineær afbildning defineret på $\Y$.
I første omgang skal vi ikke præcisere, hvordan $\X$ og $\Y$ skal
vælges. Vi skal dog forudsætte, at $\Cto \cap \C \subset \X$, således at
der at muligt at tale om $B(\ucl,\cdot)$.

\section{Fordele ved variationsformuleringen}
Når man forsøger at omformulere et differentiallignings problem til
et variations pro\-blem skyldes det naturligvis, at man opnår visse
fordele derved som fx
\begin{itemize}
  \item Vi skal som hovedregel vælge $B$, $F$, $\X$ og $\Y$ på en sådan måde, at
        problemet (V) er en generalisering af (D), dvs. (V) kan have
        løsninger, selvom (D) ikke har det.
  \item Når man søger løsninger til (D) i $\Cto \cap \C$ skyldes dette
        tra\-di\-ti\-o\-nel\-le/klas\-sis\-ke løsningsmetoder. Fysiske
        problemstillinger har dog ofte løsninger, der ikke tilhører 
        $\Cto \cap \C$, men vi er alligevel interesseret i at løse
        disse problemer. Dette er måske muligt i en mere generel formulering af
        problemet. I \cite{babuska82} findes eksempler, hvor den
        generaliserede løsning til (V) kan ses som grænseværdien for en følge af
        klassiske løsninger til (mere komplicerede) versioner af (D).
  \item Der er ofte lettere at påvise eksistens og entydighed af en
        løsning for problemer givet ved (V) fremfor (D).
  \item Fejlestimation er nemmere for problemer givet ved (V) end for (D). 
\end{itemize}  

\section{``Rimelige krav''} \label{rimeligekrav}
Som tidligere nævnt kan en omformulering af problemet (D) til (V)
konstrueres på forskellige måder. Vi skal i dette afsnit forsøge at
opstille nogle ``rimelige'' krav til bilinear formen $B$, der vil
sikre en fornuftig omformulering af problemet. Man bør skelne mellem
to typer af krav til bilinear formen $B$
\begin{enumerate}
  \item ``Fornuftige'' krav.
  \item Matematiske krav.
\end{enumerate}

\subsection{``Fornuftige'' krav}
Følgende krav bør være opfyldt for at sikre, at vores fysiske og
intuitive forestillinger om problemets opførsel holder stik
\begin{enumerate}
  \item Hvis $u \in \X$, og $B(u,v)=0$ for alle $v \in \Y$, da er
        $u=0$.
  \item Hvis der findes en løsning $\ucl$ til (D), da skal 
        $\ucl \in \X (\Omega)$, og $B(\ucl,v)=F(v)$ for alle 
        $v \in \Y$.    
  \item For en vis klasse $\mathcal{F}$ af lineære afbildninger $F$
        defineret på $\Y$, skal der for alle $F \in \mathcal{F}$,
        findes et $\uex \in \X$, så $B(\uex,v)=F(v)$ for alle  $v \in \Y$.
        Klassen $\mathcal{F}$ skal indeholde $F=0$ for at sikre, at
        det homogene problem har en løsning.  
  \item Hvis $\uex$ er en løsning til (V), gælder 
        $\|\uex\| \leq \alpha \|F\|_{\dual{\Y}}$ for alle
        $F\in\mathcal{F}$, hvor $\alpha$ er en positiv konstant uafhængig af
        $F$, og $\dual{\Y}$ er $\Y$'s duale rum. 
\end{enumerate}
\begin{remark}
Betingelse (1) følger af (4) ved at vælge $F=0$, men er nævnt som en 
selv\-stæn\-dig betingelse, idet (1) og (4) har forskellige fysiske 
fortolkninger.
\end{remark}
For at retfærdiggøre ovenstående krav skal vi nu forklare, hvilke
praktiske konsekvenser kravene har:
\begin{adenumerate}
    \item Denne betingelse sikrer entydighed af den
          generaliserede løsning. Antag at $B(u_1,v)=F(v)$, og
          $B(u_2,v)=F(v)$ for alle $v \in \Y$, da vil $B(u_1-u_2,v)=0$
          for alle $v \in \Y$. Altså må $u_1=u_2$.
    \item Denne betingelse sikrer, at når der findes en løsning
          $\ucl$ til (D), da er denne løsning også en generaliseret
          løsning til (V).
    \item Denne betingelse sikrer generaliseret eksistens og entydighed af
          variations problemet. Dette skal forstås således, at hvis
          der ændres lidt på $F$ (dvs. på data), skal der stadig
          eksistere en løsning til problemet. Hvis denne betingelse
          ikke er opfyldt, kan vi risikere, at der ikke findes en
          løsning til det generali\-se\-re\-de problem, såfremt data (dvs.
          $F$) er behæftet med en vis usikkerhed.
    \item Denne betingelse sikrer, at løsningen afhænger kontinuert af
	  data. Antag at $B(u_1,v)=F_1(v)$, og $B(u_2,v)=F_2(v)$ for
          alle $v \in \Y$, da er $B(u_1-u_2,v)=(F_1-F_2)(v)$ for
          alle $v \in \Y$, så 
          $\|u_1-u_2\| \leq \alpha \|F_1-F_2\|_{\dual{\Y}}$. Altså
          hvis $F$ ændres lidt, vil løsningen også kun ændres lidt. Hvis
          denne betingelse ikke er opfyldt, kan vi risikere at få
          ustabile numeriske løsninger, hvis $F$ fx kun er kendt
          med usikkerhed.   
\end{adenumerate}

\subsection{Matematiske krav}
For at sikre at omformuleringen fra (D) til (V) opfylder oven\-stå\-en\-de
be\-tin\-gel\-ser 1.-4., må vi stille en række matematiske krav til
bilinear formen $B$ og rummene $\X$ og $\Y$. Vi skal her præcisere
disse krav samt se, at betingelserne 1.-4. så vil være opfyldt.

Betingelsen (1) følger som nævnt tidligere af (4) ved at vælge $F=0$.

Betingelsen (2) kan man ikke sikre ved matematiske krav. Vi må derfor
være på\-pas\-se\-li\-ge, når vi udleder (V) fra (D) og holde øje med, at denne
betingelse er opfyldt.

Betingelsen (3) kan opfyldes på to forskellige måder. I begge
situationer skal vi antage, at $B$ og $F$ er begrænsede.
\begin{userenumerate}{3}
     \item Er $\X=\Y$, $\X$ et Hilbert rum, og er $B$ 
           $\X$-elliptisk, dvs. $B(u,u)\geq \alpha\norm{u}_{\X}^2$, $\forall u\in\X$, 
           hvor $\alpha$ er en positiv konstant, følger eksistens og
           entydighed af variations problemet af
           Lax-Milgrams lemma, se afsnit~\ref{eks-ent}. \label{cond31}
     \item Såfremt betingelserne 
           \begin{itemize}
             %\begin{enumerate} 
               \item $\X$ er et Banach rum og $\Y$ er et refleksivt
                     Banach rum.
               \item $\inf_{\normX{u}=1}\sup_{\normY{v}=1}|B(u,v)|\geq C>0$.
               \item For alle $v\in \Y \setminus \{0\}$ findes et $u_v\in\X$ 
                     så $|B(u_v,v)| >0$.
             %\end{enumerate}
           \end{itemize}
           er opfyldte vil eksistens og entydighed være sikret jvf.
           sætning~\ref{refleksiv} i afsnit~\ref{eks-ent}. Betingelsen~(b)
           kaldes også $\inf$-$\sup$ betingelsen eller Babu\v{s}ka-Brezzi 
           be\-tin\-gel\-sen. \label{cond32}
\end{userenumerate}
At betingelse 3.2 rent faktisk er stærkere end betingelse 3.1 er
indeholdt i følgende lemma.
\begin{lemma}
Betingelse 3.1 $\Rightarrow$ 3.2.
\end{lemma}
\begin{proof}
Da $\X =\Y$ er et Hilbert rum, er det klart at, (a) i 3.2 er opfyldt.
Da $B$ er $\X$-elliptisk vil (b) hhv. (c) i 3.2 være opfyldt ved at
vælge $u=v$ hhv. $u_v=v$.
\end{proof}
\begin{remark}
Det er klart, at 3.2 $\not\Rightarrow$ 3.1, idet der let kan gives
eksempler på Banach rum, der ikke er Hilbert rum.
\end{remark}

Betingelsen (4) kan ligeledes opfyldes på to forskellige måder
\begin{userenumerate}{4}
  \item Hvis $\X = \Y$ og $B$ er $\X$-elliptisk, dvs.  
        $B(u,u) \geq \alpha \|u\|^2_{\X}$, da vil 
        $\|\uex\|^2_{\X} \leq \frac{1}{\alpha} B(\uex,\uex) = 
        \frac{1}{\alpha} F(\uex) \leq \frac{1}{\alpha} \|F\|_{\dual{\Y}} 
        \|\uex\|_{\X}$, så
        $\|\uex\|_{\X} \leq \frac{1}{\alpha} \|F\|_{\dual{\Y}}$.
  \item Hvis der for alle $u \in \X \setminus \{0\}$, findes et 
        $v_u \in \Y \setminus \{0\}$, så en af følgende betingelser er
        opfyldt 
          \begin{enumerate}
            \item $\|v_u\|_{\Y} \leq c_1 \|u\|_{\X}$ og
                  $|B(u,v_u)| \geq c_2 \|u\|^2_{\X} \geq 
                  {\scriptstyle{\frac{c_2}{c_1}}} 
                  \|u\|_{\X} \|v_u\|_{\Y}$.
            \item $|B(u,v_u)| \geq c_3 \|u\|_{\X} \|v_u\|_{\Y}$.
          \end{enumerate}
        Da gælder 
        $\|\uex\|_{\X} \|v_u\|_{\Y} \leq \frac{1}{c} |B(\uex,v_u)| =
        \frac{1}{c} F(v_u) \leq \frac{1}{c} \|F\|_{\dual{\Y}}
        \|v_u\|_{\Y}$, dvs. \newline $\|\uex\|_{\X} \leq \frac{1}{c}
        \|F\|_{\dual{\Y}}$, hvor $c\in\{{\scriptstyle{\frac{c_2}{c_1}}},c_3\}$.      
\end{userenumerate}
\begin{remark}
Betingelse (b) i 3.2 er ækvivalent med betingelse (b) i 4.2.
\end{remark}
\begin{remark}
Rummene $\H^{p}_{k}(I), \ p\not = 1,\infty$ er refleksive.
\end{remark}

\section{Randbetingelser}
Vi skal nu se nærmere på, hvad randbetingelser betyder ved
omformuleringen til variationsproblemet. Det er svært at sige noget
om den generelle situation pga. de meget forskelligeartede
randbetingelser. Vi vil derfor nøjes med et eksempel, der illustrerer
de forhold, man skal tage højde for. Vi skal dog se, at man overordnet
(i variationsformulering sammenhæng) kan skelne mellem to typer at
randbetingelser.

\subsection{Klassificering af randbetingelser}
\begin{definition}
Randbetingelser, der kan indpasses i variationsformuleringen via
restriktioner på løsnings- og/eller testrummet, kaldes stærke (eng. essential)
randbetingelser. Randbetingelser, der ikke kan indpasses via
restriktioner, kaldes svage (eng. natural).
\end{definition}
\begin{remark}
Stærke randbetingelser kaldes også for væ\-sent\-li\-ge rand\-be\-tin\-gel\-ser,
ligesom man til tider kalder svage randbetingelser for naturlige
randbetingelser. 
\end{remark}
\begin{example}
Lad os som et eksempel betragte Poissons ligning
\begin{equation} \label{poisson}
  -\Delta u+u = f \quad \text{i $\Omega\subset\R^2$},
\end{equation}
i to dimensioner med randbetingelserne
\begin{align}
  \frac{\p u}{\p\n} &= g_1 \quad \text{på $\Gamma_1$}, \\
                 u  &= g_2 \quad \text{på $\Gamma_2$},
\end{align}
hvor $\Gamma_1\cup\Gamma_2 = \p\Omega$.
Lad os et kort øjeblik se bort fra randbetingelser. Det checkes da
let, at en klassisk løsning til \eqref{poisson} også er en løsning til
variations problemet:
\begin{equation} \label{var-poisson}
  \text{Find} \
  u\in\H^1(\dom) \ : \
  \iint_{\Omega} (\nabla u \cdot \nabla v + uv)\, dx\, dy =
  \iint_{\Omega} fv\, dx\, dy,
  \quad \forall v\in\H^1_0(\dom).
\end{equation}

Randbetingelsen $u=g_2$ på $\Gamma_2$ er en stærk randbetingelse, idet
den kan indpasses i løsningsrummet $\Het$ via restriktionen 
$u|_{\Gamma_2}=g_2$. Randbetingelsen $\afl{u}{\n}=g_1$ på $\Gamma_1$ er
derimod en svag randbetingelse (idet $\afl{u}{\n}$ ikke er defineret
på kurver i $\H^1$), så den må indpasses i selve variationsformen via leddet 
\begin{equation}
  \int_{\partial \Omega} \afl{u}{\n}v\, ds
\end{equation}
tilføjet til højresiden af~\eqref{var-poisson}. 

Vi kunne vælge at bruge samme test- og løsningsrum, men betingelsen
$v|_{\Gamma_2}=g_2$ på testrummet er upraktisk, idet det giver 
``komplicerede'' basisfunktioner. Som testfunktioner vil vi derfor vælge
de $v\in \Het$, hvor $v|_{\Gamma_2}=0$, hvilket giver os følgende 
variationsformulering af~\eqref{poisson}: Find $u\in \Het$ med $u|_{\Gamma_2}=g_2$ så
\begin{equation}
  \iint_{\Omega} (\nabla u \cdot \nabla v + uv)\, dx\, dy =
  \iint_{\Omega} fv\, dx\, dy +
  \int_{\Gamma_1} \afl{u}{\n}v\, ds
\end{equation}
for alle $v\in \Het, \ v|_{\Gamma_2}=0$

Af hensyn til teoretiske og praktiske situationer vælger man ofte samme test- og 
løsningsrum. Vi skal nu se, hvorledes dette kan opnås for ovenstående
eksempel. Sæt $u_0=u-w$, hvor $w\in{\cal C}^2(\overline{\Omega})$ er 
valgt således, at $w|_{\Gamma_2}=g_2$. Da vil $u_0|_{\Gamma_2}=0$ og
\begin{align}
  -\Delta u_0 + u_0 &= -\Delta (u-w) + (u-w) \\ 
  &=  -\Delta u + u + \Delta w - w \notag \\
  &=  f - (-\Delta w + w). \notag
\end{align}
Vi kan nu formulere variationsproblemet som: Find 
$u_0\in \Het$ med $u_0|_{\Gamma_2}=0$ så
\begin{equation}
  \iint_{\Omega} (\nabla u_0 \cdot \nabla v + u_0v)\, dx\, dy =
  \iint_{\Omega} f_0v\, dx\, dy +
  \int_{\Gamma_1} \afl{u}{\n}v\, ds
\end{equation}
for alle $v\in \Het$ med $v|_{\Gamma_2}=0$
Her er $f_0 = f -(- \Delta w +w)$.
\end{example}
Som det fremgår af definitionen og eksemplet bør man i så høj grad som
muligt ind\-passe randbetingelser som stærke randbetingelser, idet disse
har den umiddelbare fordel, at de nedsætter dimensionen af løsnings-
og/eller testrummene.

\section{Konstruktionsregler}
Man kan ikke give præcise retningslinier for, hvordan man udleder
(V) fra (D). Vi skal dog forsøge at angive nogle generelle 
konstruktionsregler for, hvordan $B$, $F$, $\X$ og $\Y$ kan
udledes, således at man får en generaliseret formulering (V) af (D).
\begin{enumerate}
  \item Udled $B$ og $F$ fra (D) udelukkende ved at bruge operationer
        som er ``i orden'' for den klassiske løsning til (D). Typiske
        eksempler er multiplikation med en testfunktioner og integration
        over det betragtede domæne.
  \item Check at  $B$ og $F$ er lineære og begrænsede.
  \item Overvej $\H^p(\Omega)$ hhv. $\H^q(\Omega)$  som løsnings- hhv.
        testrum, hvor $p$ og $q$ er valgt så tæt på hinanden som
        muligt. Derved minimeres $\max (p,q)$. Disse rum har
        flere fordele. For det første er ${\mathcal C}^p\subset\H^p$, og for
        det andet er disse rum Hilbert rum. Ved at vælge $p$ og $q$ så
        små som muligt begrænses glathedskravet til løsnings- og
        testfunktioner mest muligt. 
  \item Har differentialligningen randbetingelser, forsøg da at
        indpasse disse i form af restriktioner på løsnings- og/eller
        testrummet jvf. diskussionen om stærke randbetingelser.
  \item Check at betingelse 3 og 4 fra afsnit~\ref{rimeligekrav} er opfyldt.
\end{enumerate}

\section{Eksistens og entydigheds sætninger} \label{eks-ent}
Lad os først vise, at betingelse 3.1 fra side~\pageref{cond31} sikrer
eksistens og entydighed.
\begin{theorem}[Lax-Milgrams lemma]
Lad $\X$ være et Hilbert rum,
$B:\X\!\times\!\X\rightarrow\R$ en begrænset $\X$-elliptisk
bilinear form, og lad $F:\X\rightarrow\R$ være en begrænset lineær
afbildning. Da har variations problemet: Bestem et element $u\in\X$ så
\begin{equation} \label{app:var-prob}
B(u,v)=F(v) \qquad \forall v\in\X,
\end{equation} 
\'{e}n og kun \'{e}n løsning.
\end{theorem}
\begin{proof}
Da $B$ er begrænset findes en konstant $M$ så
\begin{equation} \label{app:lax-a-kont}
|B(u,v)|\leq M\|u\|\|v\| \qquad \text{for alle $u,v\in V$.}
\end{equation}
For ethvert $u\in\X$ er funktionalen $L:\X\rightarrow\R$ givet ved
$L(v)=B(u,v)$ kontinuert og lineær. Altså findes et entydigt bestemt
element $x_u\in\X$, se fx \cite[theorem 4.2]{rudin}, så
\begin{equation}
L(v)=\inner{v}{x_u},
\end{equation}
hvor $\inner{\cdot}{\cdot}$ er det indre produkt i $\X$. Et indre
produkt er specielt en lineær funktional, hvorfor der findes et
entydig bestemt element $f_{x_u}\in \dual{\X}$, så
$\inner{v}{x_u}=f_{x_u}(v)$,  hvor $\dual{\X}$ er det duale rum til $\X$.
Lad os for ethvert $u\in\X$ definere $Au=f_{x_u}\in\dual{\X}$, så vil
\begin{equation} \label{app:lax-Au-kont}
B(u,v)=L(v)=\inner{v}{x_u}=f_{x_u}(v)=Au(v).
\end{equation} 
Med $\|\cdot \|^{\ast}$ betegnes normen i $\dual{\X}$. Vi har nu fra
\eqref{app:lax-a-kont} og \eqref{app:lax-Au-kont}
\begin{equation}
\|Au\|^{\ast}=\sup_{v\in V\setminus\{ 0\}} 
\frac{|Au(v)|}{\|v\|}\leq M\|u\|,
\end{equation} 
hvilket viser at afbildningen $A:\X\rightarrow \dual{\X}$ er kontinuert
med $\|A\| \leq M$.

I følge Riesz' repræsentations sætning findes en afbildning 
$\tau : \dual{\X} \rightarrow \X$ så
\begin{equation}
F(v) = \inner{\tau F}{v} \qquad 
\text{for alle $F \in \dual{\X}$ og alle $v \in\X$.}
\end{equation} 
Vi har nu for $u \in\X$ og for alle $v \in\X$
\begin{equation}
B(u,v) = F(v) \Leftrightarrow
Au(v) = F(v) \Leftrightarrow
\inner{\tau Au}{v} = \inner{\tau F}{v} \Leftrightarrow
\tau Au - \tau F = 0
\end{equation}
At løse variations problemet \eqref{app:var-prob} er derfor ækvivalent
med at løse ligningen $\tau Au = \tau F$.

For at vise at denne ligning har \'{e}n og kun \'{e}n løsning, skal vi se, at
afbildningen 
\begin{equation} \label{app:kontraktion}
v\in\X \rightarrow v - \rho (\tau Av- \tau F)\in\X
\end{equation}
er en kontraktion for passende valgt $\rho$. Da $\X$ er et Hilbert rum,
er $\X$ specielt et fuldstændigt metrisk rum, og som bekendt har en
kontraktion defineret på et fuldstændigt metrisk rum netop et
fixpunkt.

Bemærkes nu at
\begin{gather}
\inner{\tau Av}{v} = Av(v) = B(v,v) \geq \alpha \|v\|^2, \\
\|\tau Av\| = \|Av\|^{\ast} \leq \|A\|\|v\| \leq M\|v\|,
\end{gather}
hvor $\alpha$ skyldes $\X$-ellipticitet, følger kontraktionen let thi
\begin{align}
\|v-\rho \tau Av\|^2 & = \|v\|^2 - 2\rho \inner{\tau Av}{v}
                         + {\rho}^2 \|\tau Av\|^2 \\
                     & \leq (1-2\rho \alpha + {\rho}^2 M^2) \|v\|^2.
\end{align}
Afbildningen defineret ved \eqref{app:kontraktion} er derfor en
kontraktion for $\rho \in ]0,2\alpha /M^2[$. Altså findes et
element $v \in V$, så $v-\rho(\tau Av - \tau F)=v$, dvs $\tau Av = \tau
F$ som ønsket.  
\end{proof}

Betingelse 3.2 fra side~\pageref{cond32} sikrer ligeledes eksistens og
entydighed jvf. nedenstående sætning
\begin{theorem} \label{refleksiv}
Lad $B:\X\!\times\!\Y\rightarrow\R$ være en begrænset bilinear form,
der opfylder betingelserne 3.2 (b) og (c) fra
side~\pageref{cond32}, og
$F:\Y\rightarrow\R$ er en begrænset lineær afbildning. Her er $\X$ et
Banach rum, og $\Y$ er et refleksivt Banach rum. Da har
variations problemet: Bestem et element $u\in\X$ så
\begin{equation} 
B(u,v)=F(v) \qquad \forall v\in\Y,
\end{equation} 
\'{e}n og kun \'{e}n løsning.
\end{theorem}
\begin{proof}
Lad $R(u):v\in\Y \rightarrow R(u)v\in\R$ være en lineær begrænset
funktional tilhørende $\dual{\Y}$ defineret ved
\begin{equation}
  R(u)v=B(u,v) \quad \forall v\in\Y.
\end{equation}
Lad tilsvarende $F\in\dual{\Y}$ være givet ved
\begin{equation}
  Fv=F(v) \quad \forall v\in\Y.
\end{equation}
Målet er at vise, at $R(\X)=\dual{\Y}$, idet vi så kan anvende
Banachs sætning, se fx \cite[s. 34]{berger}, til at få det ønskede
resultat. Lad os først vise, at $R(\X)$ er afsluttet i $\dual{\Y}$. Lad
$\{R(u_n)\}$ være en Cauchy følge i $\dual{\Y}$. Da er $\{u_n\}$ en
Cauchy følge i $\X$, thi
\begin{equation}
\begin{split}
  \|R(u_n)-R(u_m)\|_{\dual{\Y}} &= \|R(u_n-u_m)\|_{\dual{\Y}} \\
  &= \|R(\frac{u_n-u_m}{\|u_n-u_m\|_\X})\|_{\dual{\Y}}\|u_n-u_m\|_\X \\
  &= \sup_{\|v\|_{\Y=1}}|R(\frac{u_n-u_m}{\|u_n-u_m\|_\X})v|\|u_n-u_m\|_\X \\
  &\geq \inf_{\|u\|_{\X=1}}\sup_{\|v\|_{\Y=1}} |B(u,v)|\|u_n-u_m\|_\X \\
  &\geq C\|u_n-u_m\|_\X . \\
\end{split}
\end{equation}
Konstanten $C$ kommer fra $\inf$-$\sup$ betingelsen 3.2(b). Antag nu at
$\lim_{n\rightarrow\infty} u_n = u$ i $\X$. Vi finder så
\begin{equation}
\begin{split}
  \|R(u_n)-R(u)\|_{\dual{\Y}} &= \sup_{\|v\|_{\Y=1}} |B(u_n-u,v)| \\
  &\leq M\|u_n-u\|_\X , \\
\end{split}
\end{equation}  
da $B$ er begrænset. Altså vil $\lim_{n\rightarrow\infty} R(u_n) =
R(u)$ i $\dual{\Y}$, hvorfor $R(\X)$ er afsluttet i $\dual{\Y}$. Da
$R$ er lineær, vil $R(\X)$ være et afsluttet underrum af $\dual{\Y}$.

Vi vil nu vise, at $R(\X)=\dual{\Y}$. Antag omvendt at
$R(\X)\not=\dual{\Y}$. Da $R(\X)$ er et underrum af $\dual{\Y}$, vil 
$\dual{\Y}\setminus R(\X)$ indeholde mindst et underrum af dimension
1. Altså findes der et $g\in\Y^{\ast\ast}\setminus\{0\}$ så
$g(R(u))=0$ for alle $u\in\X$.

Da $\Y$ er refleksiv, er den kanoniske afbildning 
$\iota:\Y\rightarrow\Y^{\ast\ast}$ defineret ved
\begin{equation}
  \iota v(v^{\ast}) = v^{\ast}(v), \quad 
  \forall v^{\ast}\in\dual{\Y}, \ v\in\Y ,
\end{equation}  
surjektiv. Derfor findes for alle $h\in\Y^{\ast\ast}$, et $v_h\in\Y$ så
$h=\iota v_h$, dvs.
\begin{equation}
  h(v^{\ast}) = \iota v_h(v^{\ast}) = v^{\ast}(v_h) 
  \quad \forall v^{\ast}\in\dual{\Y}.
\end{equation}
Vi bemærker specielt, at 
\begin{equation}
  v_h=0\Rightarrow v^{\ast}(v_h)=0 \quad \forall v^{\ast}\in\dual{\Y}
  \Rightarrow h(v^{\ast})=0 \quad \forall v^{\ast}\in\dual{\Y} \Rightarrow h=0.
\end{equation}
Da $g\in\Y^{\ast\ast}\setminus\{0\}$ findes et
$v_g\in\Y\setminus\{0\}$ så $g(R(u))=\iota v_g(R(u))=R(u)v_g=0$
$\forall u\in\X$. Med andre ord $B(u,v_g)=0$ $\forall u\in\X$ i
modstrid med betingelse 3.2(c), hvorfor $R(\X)=\dual{\Y}$.

Vi har nu vist, at $R:\X\rightarrow\dual{Y}$ er en lineær, begrænset
og surjektiv operator, og det følger så af Banachs sætning, at $R$ har
en lineær begrænset invers $R^{-1}:\dual{\Y}\rightarrow\X$. Vælges nu
$u=R^{-1}(F)$ fås
\begin{equation}
  B(u,v)=R(R^{-1}(F))v=Fv=F(v) \quad \forall v\in\Y,
\end{equation} 
hvilket skulle vises.
\end{proof}

\section{Eksistens og entydighed for det diskrete pro\-blem}
Ovenfor har vi kun beskæftiget os med eksistens og entydighed for det
kontinuerte problem. Vi skal nu vise, at hvis man har eksistens og
entydighed for det kontinuerte problem, da vil eksistens og entydighed
af det diskrete problem følge automatisk.

Den diskrete version af problemet (V) lyder: Find $u\in\Xfe(\dom)=\Xfe$ så
\begin{equation}
  \tilde{B}(u,v) = \tilde{F}(v), \ \text{for alle $v\in\Yfe(\dom)=\Yfe$.}
\end{equation}
Her er $\tilde{B}$ og $\tilde{F}$ diskrete versioner af $B$ hhv. 
$F$, og $\Xfe$ hhv. $\Yfe$ er endelige element rum indeholdt i $\X$
hhv. $\Y$. Vi skal i næste kapitel give en formel definition af
endelige element rum. For tiden er det tilstrækkeligt at opfatte
$\Xfe$ hhv. $\Yfe$ som endelig-dimensionale underrum af $\X$ hhv.
$\Y$.  Hvis $\{\psi\}_{i=1}^{n}$ og $\{\phi\}_{j=1}^{m}$ betegner baser
for $\Xfe$ hhv. $\Yfe$, er det kendt fra den lineære algebra, at 
$\tilde{B}$ og $\tilde{F}$ kan skrives på formen
\begin{equation}
   \tilde{B}(u,v) = b^t Ka \quad \text{og} \quad \tilde{F}(v)=b^t q,
\end{equation}
hvor $u=\sum_{i=1}^{n} a_i \psi_i$, 
$v=\sum_{j=1}^{m} b_j \phi_j$, $K_{ij}=\tilde{B}(\psi_i,\phi_j)$. Vi skal
betragte tilfældet, hvor $\dim \X = \dim \Y$, dvs. $K$ er kvadratisk.
\begin{theorem} \label{eks-ent-dis}
Lad $K$ være som ovenfor. Følgende betingelser er da ækvivalente
\begin{enumerate}
  \item Det diskrete problem $\tilde{B}(u,v) = \tilde{F}(v)$ har
        \'{e}n og kun \'{e}n løsning.
  \item $K$ er invertibel.
  \item $\inf_{\|u\|_{\Xfe}=1} \sup_{\|v\|_{\Yfe}=1} |\tilde{B}(u,v)| \geq C >0$.
\end{enumerate}
\end{theorem} 
\begin{proof}
Det er klart at vi har $1. \Leftrightarrow 2.$. Lad os vise $2.
\Leftrightarrow 3.$. Vi har:
\begin{align}
 &\phantom{\Updownarrow} \inf_{\|u\|_{\Xfe}=1} \sup_{\|v\|_{\Yfe}=1} 
   |\tilde{B}(u,v)|\geq C >0 \notag \\
 &\Updownarrow \notag \\
 &\phantom{\Updownarrow} \forall u\in \Xfe\setminus\{0\} \ \exists
   v_u\in \Yfe\setminus\{0\} \ : \ |\tilde{B}(u,v_u)| \geq C
   \|u\|_{\Xfe} \|v_u\|_{\Yfe} \notag \\
 &\Updownarrow \notag \\
 &\phantom{\Updownarrow} \forall u\in \Xfe\setminus\{0\} \ \exists
   v_u\in \Yfe\setminus\{0\} \ : \ |\tilde{B}(u,v_u)| > 0 \notag \\
 &\Updownarrow \notag \\ 
 &\phantom{\Updownarrow} \forall \xi \in \R^n\setminus\{0\} 
   \ : \ K\xi \not=0 \notag \\
 &\Updownarrow \notag \\
 &\phantom{\Updownarrow} \text{$K$ er invertibel} \notag
\end{align}
\end{proof}
\begin{remark} \label{simpleremark}
Man kan vise, at betingelsen
\begin{enumerate}
  \setcounter{enumi}{3}
  \item $\inf_{\|v\|_{\Yfe}=1} \sup_{\|u\|_{\Xfe}=1} |\tilde{B}(u,v)|>0$.
\end{enumerate}
er ækvivalent med betingelserne 3. fra sætning~\ref{eks-ent-dis}.
\end{remark}
\begin{remark}
Som i det kontinuerte tilfælde, er betingelserne
\begin{gather}
  \forall u\in\X\setminus\{0\}\ \exists v_u\in\Y\setminus\{0\} \ :\  
    |\tilde{B}(u,v_u)| \geq \|u\|_{\X} \|v_u\|_{\Y} \\
  \forall v\in\Y\setminus\{0\}\ \exists u_v\in\X\setminus\{0\} \ :\  
    |\tilde{B}(u_v,v)| \geq \|u_v\|_{\X} \|v\|_{\Y} 
\end{gather}
ækvivalente med hhv. betingelse 3. fra sætning~\ref{eks-ent-dis} og
betingelse 4. fra bemærk\-setning~\ref{simpleremark}. 
\end{remark}

\section{Eksempel på konstruktion af variationsform}
\begin{example}
Som et konkret eksempel på anvendelsen af ovenstående principper skal
vi betragte følgende homogene Dirichlet problem: Bestem
$u\in\C\cap\Cto$ så
\begin{align} \label{dirichlet}
  -\Delta u+au &= f \quad \text{i $\Omega$,} \\
             u &= 0 \quad \text{på $\p\Omega=\Gamma$,} \notag
\end{align}
hvor $\Omega$ er et begrænset, sammenhængende område i $\R^2$ med
Lipschitz kontinuert rand $\Gamma$. Funktionener $a$ og $f$ antages at
tilhøre hhv. ${\mathcal L}^\infty(\Omega)$ og $\Lto$. Desuden antages
$a$ at være ikke-negativ næsten overalt i $\dom$. 

Variationsformuleringen af \eqref{dirichlet} fremkommer ved at
multiplicere \eqref{dirichlet} med en testfunktion og dernæst foretage
en partiel integration over $\Omega$. Derved fås følgende formulering
af \eqref{dirichlet}: Bestem $u\in\X$ så
\begin{equation} \label{var-dirichlet}
  B(u,v) = F(v) \quad \forall v\in\Y ,
\end{equation}
hvor 
\begin{gather}
  B(u,v)= \int_\Omega (\g u\g v +auv)\, dx \\
  \intertext{og}
  F(v) = \int_\Omega fv\, dx
\end{gather}
Jvf. ovenstående diskussion vil det her være naturligt at vælge 
$\X =\Y =\H^1_0(\Omega)$. Det er oplagt, at $B$ er symmetrisk og
bilineær, samt at $F$ er lineær. For at påvise eksistens og entydighed
af en løsning til \eqref{var-dirichlet}, vil det derfor være tilstrækkeligt at redegøre
for, at $B$ er begrænset og $\H^1_0(\Omega)$-elliptisk samt at $F$
er begrænset. Da $\X =\Y =\H^1_0(\Omega)$ er et Hilbert rum vil
eksistens og entydighed så følge af Lax-Milgrams lemma.

Lad os starte med at vise, at $B$ er begrænset. For
$u,v\in\H^1_0(\Omega)$ haves
\begin{align}
  |B(u,v)| &\leq \ssmnorm{\g u}{0}{2}\ssmnorm{\g v}{0}{2} + 
    \ssmnorm{a}{0}{\infty} \ssmnorm{u}{0}{2}\ssmnorm{v}{0}{2} \\
  &\leq \max \{ 1,\ssmnorm{a}{0}{\infty} \} 
    \ssmnorm{u}{1}{2}\ssmnorm{v}{1}{2}\ , \notag
\end{align} 
her betegner $\ssmnorm{\cdot}{0}{2}$, $\ssmnorm{\cdot}{0}{\infty}$ og
$\ssmnorm{\cdot}{1}{2}$ normerne på ${\mathcal L}^2(\Omega)$,
${\mathcal L}^\infty(\Omega)$ og $\H^1(\Omega)$. Altså er $B$
begrænset. At $B$ er $\H^1_0(\Omega)$-elliptisk følger at følgende
ulighed for $v\in\H^1_0(\Omega)$
\begin{equation}
  B(v,v) \geq \int_\Omega (\g v)^2\, dx = \sssnorm{v}{1}{2}^2.
\end{equation}
Her er $\sssnorm{\cdot}{1}{2}$ $\H^1_0$'s seminorm. Vi mangler nu kun at
vise, at $F$ er begrænset. Lad $v\in\H^1_0(\Omega)$, da vil
\begin{equation}
  |f(v)| \leq \ssmnorm{f}{0}{2} \ssmnorm{v}{0}{2} 
    \leq \ssmnorm{f}{0}{2} \ssmnorm{v}{1}{2}\ , 
\end{equation}
som ønsket

Med ovenstående har vi fået opfyldt krav 3 og 4 fra afsnit~\ref{rimeligekrav}. At
kravene 1 og 2 er opfyldt verificeres let.
\end{example}