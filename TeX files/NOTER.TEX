\chapter*{Noter}
\addcontentsline{toc}{chapter}{Noter}

Afsnittet om historien bag den endelige element metode er koncist men
ufuldst�ndigt. Dette skal forst�s s�ledes, at der ikke gives nogle
referencer til artikler, hvori ideerne er pr�senteret f�rste gang.
�rsagen til ovenst�ende valg skyldes (historiske) vanskeligheder med
at datere oprindelse for den endelige element metode samt et, fra min
side, �nske om at undg� kontroverser. For en mere fyldestg�rende
redeg�relse for historien bag den endelige element metode henvises til
artiklerne~\cite{babuska}, \cite{oden90} og \cite{zienkiewicz72}.  

Af tidsm�ssige �rsager var det desv�rre ikke muligt at implementere en
endelig ele\-ment l�ser eller dele heraf. I stedet har jeg ved at studere
programmerne FEM2D, FEMLAB-ODE~\cite{femlab-ode}, NFEARS~\cite{nfears}
og PLTMG~\cite{pltmg} f�et et indblik i, hvordan endelig element
software fungerer. Jeg har desuden haft adgang til dele af koden for
nogle af ovenn�vnte programmer og har p� den m�de haft mulighed for at
studere strukturen i s�danne programmer. 

S� godt som al eksisterende litteratur om endelige element metoder er
engelsk sproget. Dette giver af og til vanskeligheder med at finde
passende danske udtryk. I de situationer hvor det har v�ret specielt
sv�rt, har jeg valgt at angive det engelske udtryk i parentes efter
det danske udtryk. En del
af eksemplerne i dette speciale tager udgangspunkt i fysikkens verden.
Min ikke overv�ldende fysiske indsigt g�r, at jeg ligeledes her har valgt at
angive det engelske fagudtryk i parentes efter det danske udtryk. P� den
m�de skulle eventuelle un�jagtigheder kunne modvirkes. 