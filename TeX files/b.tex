\documentstyle[latin1]{article}
\newcommand{\dom}{\Omega}
\newcommand{\mathcal}{\cal}
\newcommand{\eqref}[1]{(\ref{#1})}
\newcommand{\I}{{\mathcal I}}
\newcommand{\T}{{\mathcal T}}
\newcommand{\D}{{\mathcal D}}
\begin{document}
Målet med dette kapitel er at præsentere en netinddelings 
metode, der giver asymptotiske optimale netinddelinger 
(indenfor klassen af netinddelinger med rektangulære 
elementer). Metoden bygger på begrebet net tætheds funktion 
og blev indført af Babu\v{s}ka og Gui i~\cite{}. Metoden er 
senere videreudviklet af Hugger i artiklerne~\cite{}, 
\cite{}, \cite{}, \cite{}, \cite{} og \cite{}. Med net 
tætheds funktioner kan man beskrive følger af netinddelinger
ved hjælp af simple midler, og man undgår dermed de 
komplicerede datastrukturer, der ofte karakteriserer 
netinddelings algoritmer. Ønsket om simple metoder til at 
beskrive netinddelinger på udspringer fra parametriserede
problemer, hvor man med fordel ofte kan genanvende tidligere
netinddelinger. Det vil fremgå af teorien nedenfor, at 
metoden med fordel også kan anvendes, selvom man kun skal 
løse et enkelt problem. Netinddelings algoritmen er ment som 
en ``one shot'' metode, der på baggrund af informationer fra 
net tætheds funktionen og den bruger bestemte fejltolerance 
forsøger at konstruere en netinddeling, hvor fejlen i den
tilhørende endelige element løsning efter første gennemløb af 
algoritmen er mindre end den givne fejltolerance. Mislykkes 
dette, vil metoden iterere.
 
% \begin{defintion}
Lad $\T=\{\Box_j\}_{j=1}^m$ være et endeligt element net. Vi
siger da, at $\T$ er en optimal endelig element netinddeling
indenfor en given klasse af netinddelinger, såfremt at fejlen
i den tilhørende endelige element løsning er mindre end en 
given fejltolerance, og såfremt de beregningsmæssige 
omkostninger ved at finde denne løsning er minimale. Minumum er 
her taget over alle netinddelinger indenfor den givne klasse.
% \end{defintion}

% \begin{remark}
I algoritme~\ref{} udføres forgrovning og forfining kun en 
enkelt gang. Dette er valgt for at undgå muligheder for 
uendelige løkker. Forgrovning udføres før forfining for at sikre
en mere effektiv udnyttelse af computer lagerplads. Ved at udføre
forgrovning og forfining i nævnte rækkefølge kan man nemlig 
udnytte den lagerplads, der er blevet frigjort i forbindelse med
forgrovning til at gemme informationer om forfining i. For flere 
detaljer om algoritmen henvises til afsnit~\ref{}, hvor vi 
konstruerer en mere generel algoritme. Den initiale netinddelingen
$\hat{\T}$ er bruger bestemt og kan eventuelt vælges som $\T^{\ast}$.
% \end{remark}

% \begin{proof}
Det er klart, at $A_n^{\ast}(\D)$ er en tilladt ikke-aftagende
netinddeling af $\dom$ for enhver net tætheds funktion $\D$. Dette
følger af den måde, hvorpå netinddelinger i $A_n^{\ast}(\D)$ er 
sorteret samt af, at algoritmen~\ref{} er konstueret således, at 
ethvert element på et eller andet tidspunkt bliver inddelt i mindre
elementer når intensiteten aftager. Altså vil det samlede antal 
elementer gå mod uendelig, når intensiteten går mod nul. For at vise 
at det resterende udsagn i sætningen er korrekt, skal vi vise, at 
$\D$ defineret ved~\eqref{} og \eqref{} opfylder 
$A_n^{\ast}(\D) = \{\T_m\}_{m=1}^\infty$, hvor $\T_m$ er netinddelingen
givet ved algoritme~\ref{}. Da $\{\T_m\}_{m=1}^\infty$ antages at 
tilhøre billedet af $A_n^{\ast}$, findes der en net tætheds funktion
$\tilde{\D}$ så $A_n^{\ast}(\tilde{\D})=\{\T_m\}_{m=1}^\infty$. Lad
$S_0$ betegne den mængde, hvor $\tilde{\D}$ er diskontinuert eller 
har singulariteter. Mængden $S_0$ har mål $0$. Bemærk at mængden 
$S\equiv S_0 \cup ( \cup_{m=1}^\infty \cup_{j=1}^{N_{a,m}} \partial
\Box_{j,m})$ ligeledes har mål $0$. Sæt nu
$a_m=\int_{\Box_m(x)} \tilde{\D}\, dx$ og lad $\I_m$ være et 
vilkårligt tal. Ved at indsætte $\I_m$ for $\I$ i algoritmen~\ref{}
fås en netinddeling $\{\T_m\}_{m=1}^\infty$. Ifølge algorimen vil en
forfining af $\Box_m(x)$ kun ske såfremt $a_m(x)/\I_m > \alpha$
Omvendt følger det af konstruktionen af algoritmen, at 
$a_m(x)/\I_m \leq \alpha$. Fra defintionen af indeks delfølgen
$\{ m_k \}_{k=1}^\infty$ følger det af $a_{m_k}/\underline{\I}_{m_k}
= \alpha$ for alle positive heltal $k$. Vi har nu
\begin{equation}
  \tilde{\D}(x) = \lim_{m\rightarrow\infty}
    \frac{a_{m_k}(x)}{V_n(\Box_{m_k}(x))}
\end{equation}
\begin{equation}
  = \frac{a_{m_k}(x)}{\underline{\I}_m} 
    \lim_{m\rightarrow\infty} 
    \frac{\underline{\I}_m}{V_n(\Box_{m_k}(x))}
\end{equation}
\begin{equation}
  = \alpha \frac{\underline{\I}_m}{V_n(\Box_{m_k}(x))}
\end{equation}
Da $S$ har mål $0$ følger det af~\eqref{}, at $\D$ givet ved~\eqref{} 
er den rigtige net tætheds funktion, da den er identiske med $\tilde{\D}$
næsten overalt.
% \end{proof}
\setlength{\unitlength}{1mm}
\begin{figure}[htb]
\begin{center}
\begin{picture}(100,40)(0,0)
\put(0,2.5){\begin{picture}(40,35)(0,0)
\put(0,0){\line(1,0){40}}
\put(0,0){\line(3,5){20}}
\put(40,0){\line(-3,5){20}}
% \dashline{2}(20,12.5)(10,16.67)
% \dashline{2}(20,12.5)(30,16.67)
% \dashline{2}(20,12.5)(20,0)
\end{picture}
}
\put(60,0){\begin{picture}(40,40)(0,0)
\put(0,0){\line(1,0){40}}
\put(0,0){\line(0,1){40}}
\put(0,40){\line(1,0){40}}
\put(40,0){\line(0,1){15}}
\put(40,40){\line(0,-1){15}}
\put(20,20){\line(4,1){20}}
\put(20,20){\line(4,-1){20}}
% \dashline{2}(20,20)(0,20)
% \dashline{2}(20,20)(20,0)
%\dashline{2}(20,20)(20,40)
\end{picture}
}
\end{picture}
\end{center}
\caption{Eksempler på $\beta$-domæner}
\end{figure}
\end{document}






