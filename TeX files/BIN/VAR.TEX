\documentstyle[latin1]{report}
\newtheorem{definition}{Definition}
\newtheorem{theorem}{S�tning}
\newtheorem{proof}{Bevis}
\newtheorem{remark}{Bem�rkning}
% \newenvironment{note}{\begin{remark}}{\hfill a \end{remark}}
\newcommand{\Th}{{\cal{T}}_h}
\newcommand{\Nh}{{\cal{N}}_h}
\newcommand{\NK}{{\cal{N}}_K}
\newcommand{\Het}{H^1(\Omega)}
\newcommand{\Hto}{H^2(\Omega)}
\newcommand{\Lto}{L^2(\Omega)}
\newcommand{\R}{{\bf R}}
\newcommand{\fe}{(K,P,\Sigma)}
\newcommand{\hatfe}{(\hat{K},\hat{P},\hat{\Sigma})}
\newcommand{\familyfe}{(K,P_K,\Sigma_K)}
\newcommand{\afl}[2]{\frac{\partial #1}{\partial #2}}
\newcommand{\domain}{\overline{\Omega}}
\newenvironment{Thenumerate}{%
  \renewcommand{\labelenumi}{${\cal T}_{h}$\theenumi.}%
  \begin{enumerate}}{\end{enumerate}}
\begin{document}
Lad os et kort �jeblik se bort fra randbetingelser. Det checkes da
let aten l�sning til [] ogs� er en l�sning til variations problemet
\begin{equation}
  \int_{\Omega} (\nabla u \cdot \nabla v + uv)dxdy =
  \int_{\Omega} fvdxdy
\end{equation}
og omvendt.

Randbetingelsen $u=g_2$ p� $\Gamma_2$ er en st�rk randbetingelse, idet
den kan indpasses i l�sningsrummet $\Het$ via restriktionen 
$u|_{\Gamma_2}=g_2$. Randbetingelsen $\afl{u}{n}=g_1$ p� $\Gamma_1$ er
derimod en svag randbetingelse (idet u ikke er differentiabel), s� den
m� indpasses i selve variationsformen via leddet 
\begin{equation}
  \int_{\partial \Omega} \afl{u}{n}vds
\end{equation}
tilf�jet til h�jresiden af 

Vi kunne nu v�lge at bruge samme test- og l�sningsrum, men betingelsen
$v|_{\Gamma_2}=g_2$ p� testrummet er upraktisk, thi den giver 
``komplicerede'' basisfunktioner. Som testfunktioner vil vi derfor v�lge
de $v\in \Het$, hvor $u|_{\Gamma_2}=0$, hvilket giver os f�lgende 
variationsformulering af []: Find $u\in \Het, \ u|_{\Gamma_2}=g_2$ s�
\begin{equation}
  \int_{\Omega} (\nabla u \cdot \nabla v + uv)dxdy =
  \int_{\Omega} fvdxdy +
  \int_{\Gamma_1} \afl{u}{n}vds
\end{equation}
for alle $v\in \Het, \ u|_{\Gamma_2}=0$

Af hensyn til praktiske situationer v�lger man ofte samme test- og 
l�sningsrum. Vi skal nu se, hvorledes dette kan opn�s for ovenst�ende
eksempel. S�t $u_0=u-w$, hvor $w\in{\cal C}^2(\overline{\Omega})$ er 
valgt s�ledes at $w|_{\Gamma_2}=g_2$. Da vil $u_0|_{\Gamma_2}=0$ og
\begin{equation}
  -\Delta u_0 + u_0 = -\Delta (u-w) + (u-w) = 
  -\Delta u + u + \Delta w - w = 
  f - (-\Delta w + w)
\end{equation}
Vi kan nu formulere variationsproblemet som: Find 
$u_0\in \Het, \ u_0|_{\Gamma_2}=0$ s�
\begin{equation}
  \int_{\Omega} (\nabla u_0 \cdot \nabla v + u_0v)dxdy =
  \int_{\Omega} f_0vdxdy +
  \int_{\Gamma_1} \afl{u}{n}vds
\end{equation}
for alle $v\in \Het, \ u|_{\Gamma_2}=0$
Her er $f_0 = f + \Delta w -w$

\section{Eksistens og entydighed for det diskrete problem}
Den diskret version af problemet (V) lyder: Find $u\in U_h$ s�
\begin{equation}
  \tilde{B}(u,v) = \tilde{F}(v), \ for alle v\in V_h
\end{equation}
Her er $\tilde{B}$ og $\tilde{F}$ den diskrete version af $B$ hhv. 
$F$. Hvis $\{psi\}_{i=1}^{n}$ og $\{phi\}_{j=1}^{m}$ betegner baser
for $U_h$ hhv. $V_h$ er det kendt fra den line�re algebra, at 
$\tilde{B}$ og $\tilde{F}$ kan skrives p� formen
\begin{equation}
   \tilde{B}(u,v) = b^t Ka, \ \ \ \tilde{F}(v)b^t q
\end{equation}
hvor $u=\sum_{i=1}^{n} a_i \psi_i$, 
$v=\sum_{j=1}^{m} b_j \phi_j$, $K_{ij}=k(\psi_i,\phi_j)$. 

Vi skal kun betragte tilf�ldet, hvor $\dim U = \dim V$, dvs. $K$ er
kvadratisk.
\begin{theorem}
Lad $K$ v�re som i []. F�lgende betingelser er da �kvivalente
\begin{enumerate}
  \item Det diskrete problem $\tilde{B}(u,v) = \tilde{F}(v)$ har
        en og kun en l�sning.
  \item $K$ er invertibel.
  \item $\inf_{\|u\|_U=1} \sup_{\|v\|_V=1} |\tilde{B}(u,v)| \geq C >0$.
\end{enumerate}
\end{theorem} 
\begin{proof}
Det er klart at vi har $1. \Leftrightarrow 2.$. Lad os vise 
\vspace{2mm}
$2. \Leftrightarrow 3.$. Vi har:

$\inf_{\|u\|_U=1} \sup_{\|v\|_V=1} |\tilde{B}(u,v)|\geq C >0$.

$\Updownarrow$

$\forall u\in U\setminus\{0\} \exists v_u\in V\setminus\{0\}
 \ : \ |\tilde{B}(u,v)| \geq C \|u\|_U \|v_u\|_V$

$\Updownarrow$

$\forall \xi \in \R^n\setminus\{0\} \ : \ K\xi \not=0$

$\Updownarrow$

$K$ er invertibel.
\end{proof}
\begin{remark}
Man kan vise at betingelsen
\begin{enumerate}
  \setcounter{enumi}{3}
  \item $\inf_{\|v\|_V=1} \sup_{\|u\|_U=1} |\tilde{B}(u,v)|>0$.
\end{enumerate}
er �kvivalent med betingelserne 1-3 fra s�tning []
\end{remark}

\section{Valg af {\cal U} og {\cal V}} 

\section{Konstruktionsregler}
Man kan ikke give pr�cise retningslinier for, hvordan man udleder
(V) fra (D). Vi skal dog fors�ge at angive nogle generelle 
konstruktionsregler for, hvordan $B$, $F$, $\cal U$ og $\cal V$ kan
udledes, s�ledes at man f�r en generaliseret formulering (V) af (D).
\begin{enumerate}
  \item Udled $B$ og $F$ fra (D) udelukkende ved at bruge operationer
        som er ``i orden'' for den klassiske l�sning til (D). Typiske
        eksempler er multiplikation med en testfunktioner og integration
        over det betragtede dom�ne.
  \item Check at  $B$ og $F$ er line�re og begr�nsede.
  \item Overvej $H^p_k(\Omega)$ som l�snings- og testrum.
  \item Check at betingelse 3 og 4 fra afsnit [] er opfyldt.
\end{enumerate}

\end{document}









